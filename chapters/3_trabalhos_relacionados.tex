\phantomsection

\chapter{Trabalhos relacionados}\label{chapter:trabalhos-relacionados}
\phantomsection

Neste capítulo, vamos explorar outras linguagens usadas para desenvolvimento \textit{backend},
construindo uma aplicação simples em cada uma delas. O objetivo é comparar funcionalidades,
padrões de projeto, sintaxe das linguagens, e outras características. Para selecionar
as linguagens, vamos nos basear em índices de uso e satisfação de linguagens. Cada
linguagem será apresentada em uma subseção, e os critérios de escolha das linguagens
serão especificados na subseção que as apresenta.

\section{Aplicação base}\label{sec:aplicacao-base}

Para comparar as linguagens, vamos construir uma aplicação simples, que será usada
como base para as comparações. A aplicação consiste em um sistema de cadastro de
usuários (um simples \textit{CRUD} -- \textit{Create, Read, Update, Delete}). A
aplicação deve receber requisições HTTP para cadastrar, atualizar, deletar e listar
usuários. Os dados devem ser enviados através do corpo da requisição, em formato
JSON, seguindo a convenção REST. Cada usuário deve ter um id numérico (começando
em 0, e aumentando em 1 a cada usuário cadastrado), nome e idade. A
\autoref{tab-base-endpoints} apresenta os \textit{endpoints} da aplicação, onde
\texttt{/users/:id} significa uma rota que possui um número após \texttt{/users/},
como em \texttt{/users/1}. Este número equivale ao id numérico de um usuário.


\begin{table}[htb]
  \caption[\textit{Endpoints} da aplicação base]{\textit{Endpoints} da aplicação base}
  \label{tab-base-endpoints}
  \resizebox{\textwidth}{!}{%
    \begin{tabular}{p{2.85cm}p{2.85cm}p{8.55cm}}
      \toprule
      \textbf{Método HTTP} & \textbf{\textit{Endpoint}} & \textbf{Descrição}                                                                         \\
      \midrule
      GET                  & /users                     & Lista todos os usuários cadastrados                                                        \\
      GET                  & /users/:id                 & Lista um usuário específico                                                                \\
      POST                 & /users                     & Cria um novo usuário. O nome e a idade devem ser enviados no corpo da requisição           \\
      PUT                  & /users/:id                 & Atualiza um usuário específico. O nome e a idade devem ser enviados no corpo da requisição \\
      DELETE               & /users/:id                 & Deleta um usuário específico                                                               \\
      \bottomrule
    \end{tabular}
  }
\end{table}

\section{Linguagens relacionadas}

Nesta seção, vamos construir a aplicação base apresentada na \autoref{sec:aplicacao-base}
em algumas linguagens de programação, a fim de comparar as linguagens.

Além de escolher uma linguagem, também precisaremos escolher \textit{frameworks}
para construir a aplicação. Para cada linguagem, vamos procurar por \textit{frameworks}
utilizando a pesquisa do GitHub, pesquisando \textbf{web framework language:\textit{linguagem}},
e escolhendo o resultado com maior número de estrelas.

\subsection{Python}\label{subsec:python}

De acordo com \textcite{tiobeindex}, Python é a linguagem de programação mais procurada
no mundo, além de ter ganho o prêmio linguagem do ano nos anos 2007, 2010, 2018,
2020 e 2021 pelo mesmo índice.

Criada nos anos 80, é usada em diversas áreas, como ciência de
dados, desenvolvimento web, desenvolvimento de jogos, entre outras. Python é uma
linguagem de programação interpretada, de alto nível, de tipagem dinâmica e
multi-paradigma (\textcite{pythonmanual}).

Os três maiores \textit{frameworks} em termo de estrelas no GitHub são Django,
Flask e FastAPI, nesta ordem. Como Django é um \textit{framework fullstack} (ou seja,
ele é um \textit{framework} que abrange tanto o desenvolvimento \textit{frontend}
quanto o desenvolvimento \textit{backend}, e não é usado para APIs REST), vamos
usar o Flask. O \autoref{code:flaskapp} apresenta a aplicação base desenvolvida
em Python com Flask.

\begin{lstlisting}[language=Python,label={code:flaskapp},caption={Aplicação base em Python com Flask}]
from flask import Flask, jsonify, request, Response

app = Flask(__name__)
users = {}
last_index = 0

@app.route('/users', methods=["GET"])
def index():
    return jsonify(users)

@app.route('/users/<int:user_id>', methods=["GET"])
def show(user_id):
    return jsonify(users[user_id])

@app.route('/users', methods=['POST'])
def create():
    global last_index

    body = request.get_json()
    last_index = last_index + 1
    users[last_index] = body

    return jsonify({'id': last_index})

@app.route('/users/<int:user_id>', methods=["PUT"])
def update(user_id):
    body = request.get_json()
    users[user_id] = body

    return Response(status=200)

@app.route('/users/<int:user_id>', methods=["DELETE"])
def delete(user_id):
    users.pop(user_id)

    return Response(status=200)

\end{lstlisting}

Flask se baseia no uso de \textit{decorators} para definir as diferentes rotas
da aplicação. Isso faz com que seja rápido e fácil de escrever programas com Flask,
mas também pode ser confuso para quem não está acostumado com o uso de \textit{decorators}
ou do \textit{framework} em si.

\subsection{C\#}

C\# é a próxima linguagem comumente usada para aplicações \textit{backend} depois
de Python e Java (\textcite{tiobeindex}). C\# foi escolhida ao invés de Java por
serem duas linguagens bastante similares, e C\# ser mais procurada, mais amada,
e menos temida por desenvolvedores do que Java (\textcite{stackoverflowsurvey}).

C\# foi desenvolvida por Anders Hejlsberg, na Microsoft, em 2000, com o intuito
de ser uma linguagem simples, moderna, orientada a objetos e com segurança de tipos
para a plataforma .NET (\textcite{csharpmanual}).

A plataforma .NET possui o \textit{framework} ASP.NET, feito para desenvolver
aplicativos web e serviços. Essa é a opção mais popular para desenvolvimento web
com C\#, e é a que vamos usar. O \autoref{code:aspnetapp} apresenta parte da implementação
da aplicação base em C\# com ASP.NET, mostrando apenas a rota de atualizar um
usuário.

\begin{lstlisting}[language=C++,label={code:aspnetapp},caption={Aplicação base em C\# com ASP.NET}]
[Route("users")]
[ApiController]
public class UserController : ControllerBase
{
    ...

    [HttpPut("{id}")]
    public async Task<IActionResult> UpdateUser(long id, UserInput input)
    {
        var user = await _context.FindAsync<User>(id);
        if (user == null)
        {
            return NotFound();
        }
        user.Name = input.Name;
        user.Age = input.Age;
        _context.Entry(user).State = EntityState.Modified;
        await _context.SaveChangesAsync();

        return NoContent();
    }

    ...
}
\end{lstlisting}

Como C\# é uma linguagem fortemente orientada a objetos, o \textit{framework} ASP.NET
faz forte uso classes, objetos e herança, além de também usar \textit{decorators}.
O corpo das requisições é obtido automaticamente como um parâmetro do método que
responde à requisição, o que também pode ser confuso.

Por outro lado, o \textit{framework} tem boas ferramentas de geração de código,
que geram boa parte do código básico necessário para uma aplicação, o que pode
agilizar o desenvolvimento inicial.

Um simples projeto necessita de vários arquivos de configuração (gerados automaticamente),
que também podem ser confusos para quem não está acostumado com o \textit{framework}.


\subsection{Javascript}\label{subsec:js}

Javascript é a próxima linguagem comumente usada para aplicações \textit{backend}
depois de C\# na tabela \textcite{tiobeindex}, e é a linguagem alvo para Ipe.

Originalmente desenvolvida por Brendan Eich na Netscape em 1995 para adicionar
interações em páginas web, Javascript é uma linguagem de script interpretada, de
tipagem dinâmica e fracamente tipada, e orientada a objetos, com alguns aspectos
de programação funcional. Javascript vem se tornando cada vez mais popular em
outros meios, como o \textit{backend}, dispositivos \textit{Internet of Things}
(\textit{IOT}), microsserviços, aplicações desktop, entre outros. Por conta de
sua popularidade, muitas bibliotecas, \textit{frameworks} e ferramentas existem
para auxiliar no desenvolvimento de aplicações Javascript.  Notavelmente, existem
esforços para adicionar tipagem estática, principalmente através do Typescript,
que já é mais amado, mais procurado e menos temido do que o Javascript
\textcite{stackoverflowsurvey}.

Para desenvolver a aplicação base, usaremos a plataforma Node.js, que possibilita
executar programas Javascript fora do navegador, e o \textit{framework} Express.
O \autoref{code:nodeapp} apresenta a implementação da aplicação base em Javascript.


\begin{lstlisting}[language=Javascript,label={code:nodeapp},caption={Aplicação base em Javascript com Express}]
const express = require("express");

const app = express();

const users = new Map();
let lastId = 0;

app.use(express.json());

app.get("/users", (req, res) => {
  const allUsers = Array.from(users.values());

  res.json(allUsers);
});

app.get("/users/:userId", (req, res) => {
  const userId = parseInt(req.params.userId);
  const user = users.get(userId);

  res.json(user);
});

app.post("/users", (req, res) => {
  const { name, age } = req.body;
  lastId++;
  users.set(lastId, { name, age, id: lastId });

  res.json({ id: lastId });
});

app.put("/users/:userId", (req, res) => {
  const userId = parseInt(req.params.userId);
  const { name, age } = req.body;
  users.set(userId, { name, age, id: userId });

  res.json({ id: userId });
});

app.delete("/users/:userId", (req, res) => {
  const userId = parseInt(req.params.userId);
  users.delete(userId);

  res.json({ id: userId });
});

app.listen(3000, (err) => {
  if (err) {
    console.log(err);
  }

  console.log("Server is running on port 3000");
});
\end{lstlisting}

O \textit{framework} Express usa algumas ideias de programação funcional, como
funções de alta ordem e funções anônimas, o que gera um código bastante legível
e expansível. Além disso, tudo é bastante explícito, sem \textit{decorators} ou
geração de código, o que também melhora a legibilidade, com baixo custo em velocidade
de desenvolvimento.

A abstração de rotas é feita através de funções, que recebem como parâmetros
um objeto representando a requisição HTTP e outro representando a resposta. Uma
abstração melhor seria receber apenas a requisição como parâmetro, e retornar uma
resposta.

\subsection{Elixir}

Elixir é a segunda linguagem mais amada, mais procurada e menos odiada, perdendo
apenas para Rust (\textcite{stackoverflowsurvey}). Rust não foi escolhido por que
a abordagem dos \textit{frameworks} em Rust são parecidas com as abordagens já
vistas nas linguagens anteriores. Outro ponto para escolher Elixir é o fato de
Elixir ser uma linguagem funcional, ou seja, uma linguagem diferente das outras
já vistas, e similar a Ipe.

Elixir é uma linguagem de programação funcional, concorrente e de tipagem dinâmica,
desenvolvida por José Valim em 2012. Elixir é compilada para a máquina virtual
BEAM, que é concorrente e tolerante a falhas, e também é usada pela linguagem
Erlang.

Para desenvolver a aplicação base, usaremos o \textit{framework} Phoenix.

O \textit{framework} Phoenix se aproveita bastante da geração automática de código
e da funcionalidade de \textit{macros} do Elixir, o que gera um código mais conciso,
mas muito mais difícil de entender. A aplicação inteira foi gerada automaticamente,
o que facilita e agiliza o início de um projeto. Porém, isso dificulta na manutenção
e compreensão do código.

Como a ferramenta gera bastante código, apenas algumas partes são apresentadas a
seguir.

O \autoref{code:elixirrouter} apresenta a definição das rotas da aplicação, que
permite definir funções a serem executadas antes de cada rota, além do agrupamento
de rotas.


\begin{lstlisting}[language=Elixir,label={code:elixirrouter},caption={Definição das rotas da aplicação em Elixir com Phoenix}]
pipeline :api do
  plug(:accepts, ["json"])
end

scope "/users", UserCrudWeb do
  pipe_through(:api)
  get("/", UserController, :index)
  get("/:id", UserController, :show)
  post("/", UserController, :create)
  put("/:id", UserController, :update)
  delete("/:id", UserController, :delete)
end
\end{lstlisting}

O \autoref{code:elixircreate} apresenta a implementação do método de criação de
usuários. O uso do operador \texttt{|>} (lê-se \textit{pipe}) permite a composição
de funções, o que gera um código mais conciso, e descreve uma sequência de
transformações de dados, um dos pilares da programação funcional. Também podemos
ver que a função recebe um mapa como parâmetro, que representa o corpo da requisição
HTTP.

\begin{lstlisting}[language=Elixir,label={code:elixircreate},caption={Função de criação de usuário em Elixir com Phoenix}]
def create(conn, %{"user" => user_params}) do
  with {:ok, %User{} = user} <- Accounts.create_user(user_params) do
    conn
    |> put_status(:created)
    |> put_resp_header("location", Routes.user_path(conn, :show, user))
    |> render("show.json", user: user)
  end
end
\end{lstlisting}

\subsection{Haskell}\label{subsec:haskell}

Haskell é a linguagem número 50 na tabela \textcite{tiobeindex}, e foi escolhida
por ser considerada a principal linguagem funcional hoje em dia.

Haskell é uma linguagem de programação de propósito geral, puramente funcional,
que inclui funções de alta ordem, tipos polimórficos, inferência de tipos e
tipagem estática (\textcite{conceptionoffunctionalpl}).

O \textit{framework} escolhido foi o IHP, que, similarmente ao Phoenix do Elixir,
usa ferramentas de geração de código para gerar a maior parte do código necessário
de uma aplicação. Ao invés de usar \textit{macros}, o IHP usa tipos polimórficos
e outros recursos da linguagem para diminuir repetição de código. Haskell é
famosa por ser difícil de aprender e usar, e o uso extensivo de recursos avançados
da linguagem tornam o código gerado pelo IHP difícil de entender.

O \autoref{code:haskellcreate} apresenta a implementação do método de criação de
usuários. Em Haskell, também é comum vermos a estrutura de composição de funções
(com o operador \texttt{|>}), assim como vimos em Elixir.

\begin{lstlisting}[language=Haskell,label={code:haskellcreate},caption={Função de criação de usuário em Haskell com IHP}]
action CreateUserAction = do
    let user = newRecord @User
    user
        |> buildUser
        |> ifValid \case
            Left user -> render NewView { .. } 
            Right user -> do
                user <- user |> createRecord
                setSuccessMessage "User created"
                redirectTo UsersAction
\end{lstlisting}

\section{Conclusões dos experimentos}

Agora que experimentamos diferentes linguagens, com diferentes paradigmas, podemos
recolher aprendizados e conclusões sobre o que foi visto. Nesta seção, discutiremos
alguns pontos vistos como importantes para a linguagem Ipe. Além de analisar cada
ponto nas linguagens vistas anteriormente, vamos discutir as lições aprendidas e
usadas para o desenvolvimento de Ipe.

A \autoref{tab-language-comparisons} apresenta uma comparação entre as
principais características das linguagens vistas, além do que teremos em Ipe. As
subseções seguintes discutem alguns aspectos gerais das linguagens em maiores detalhes.

\begin{table}[htb]
  \caption{Principais características das linguagens analisadas e de Ipe}
  \label{tab-language-comparisons}
  \resizebox{\textwidth}{!}{%
    \begin{tabular}{p{2.50cm}p{4.00cm}p{2.50cm}p{5.25cm}}
      \toprule
      \textbf{Linguagem} & \textbf{Paradigma principal}             & \textbf{Tipagem} & \textbf{Nível de temor \cite{stackoverflowsurvey}} \\
      \midrule
      Python             & OO                                       & Dinâmica         & 32,66\%                                            \\
      C\#                & OO                                       & Estática         & 36,61\%                                            \\
      Javascript         & OO com protótipos e elementos funcionais & Dinâmica         & 38,54\%                                            \\
      Elixir             & Funcional                                & Dinâmica         & 24,54\%                                            \\
      Haskell            & Puramente funcional                      & Estática         & 43,56\%                                            \\
      Ipe                & Puramente funcional                      & Estática         & --                                                 \\
      \bottomrule
    \end{tabular}
  }
\end{table}

Podemos ver na \autoref{tab-language-comparisons} que Javascript possui um alto nível de temor (38,54\%). Isso
provavelmente se dá pelo fato do seu sistema de tipos ser dinâmico e fraco, pois,
na mesma pesquisa, o nível de temor de Typescript é 12 pontos percentuais menor
(26,54\%). Haskell é extremamente temida, mas ao mesmo tempo, é a 16\textordfeminine{}
linguagem mais desejada. Estima-se que isso seja culpa da grande complexidade da
linguagem, e sua curva de aprendizado íngreme -- muitas pessoas querem conhecer
a programação funcional, mas encontram grande dificuldade em aprender Haskell.
Elixir está apenas uma posição acima de Haskell no índice de linguagens mais
desejadas. Porém, como é uma linguagem funcional com uma barreira de entrada menor,
é a segunda linguagem mais amada e menos temida. Ipe almeja ser uma linguagem
puramente funcional, com tipagem forte e estática, como Haskell, mas com uma
curva de aprendizado mais suave, como Elixir, Javascript e Python. Desta maneira,
desenvolvedores poderão conhecer a programação funcional fortemente tipada com
uma barreira de entrada menor. Afinal, um dos objetivos de Ipe é ajudar a
ensinar programação funcional com um bom sistema de tipos.


\subsection{Simplicidade e funcionalidades}

Linguagens mais maduras ou mais antigas tendem a ter várias funcionalidades, e
cada funcionalidade pode ter mais de uma forma de ser usada. Por exemplo, em
Javascript uma variável pode ser definida com \texttt{var}, \texttt{let} ou
\texttt{const}, e funções podem ser criadas com a palavra reservada \texttt{function}
ou em formato de \textit{arrow function}. Isso pode ser confuso para quem está
aprendendo a linguagem, além de gerar divisões entre os usuários da linguagem.

Outro exemplo de alta complexidade é Haskell, que possui inúmeras extensões de
linguagem, que servem para melhorar recursos da linguagem, ou adicionar novos
recursos. Pode ser difícil mudar de uma base de código que usa certas extensões
de linguagem para outra base de código que não as usa. Mesmo sendo a mesma linguagem,
a forma de usá-la pode ser extremamente diferente em cada base de código.

Como Ipe é uma linguagem específica para aplicações \textit{backend}, podemos
montar uma linguagem mais simples, com menos funcionalidades. Assim, a curva de
aprendizado deve ser menor, e a linguagem deve ser mais fácil de usar. Além disso,
código Ipe deve ser padronizado -- cada tarefa deve ter uma única forma de ser feita --,
de forma que não leve muito tempo para entender um projeto novo, vindo de outro projeto.
Neste sentido, podemos utilizar de alguns pontos da filosofia do Python:

\begin{itemize}
  \item \textit{Explicit is better than implicit.}
  \item \textit{There should be one-- and preferably only one --obvious way to do it.}
\end{itemize}

\cite{zenofpython}

Que, em tradução livre, significam:

\begin{itemize}
  \item \textit{Explícito é melhor do que implícito.}
  \item \textit{Deveria existir uma-- e de preferência apenas uma --maneira óbvia de se fazer algo.}
\end{itemize}

Também notamos que o uso de \textit{macros}, \textit{decorators} e tipos polimórficos
dificulta a compreensão do código, pois gera mais código implícito. Ipe tem o objetivo
de ser uma linguagem extremamente explícita -- mesmo que isso custe mais linhas
de código para usuários da linguagem.

\subsection{Operadores}

A composição de funções através de \textit{pipelines} com o operador
\texttt{|>} produz código conciso que expressa muito bem a ideia de transformações
sucessivas de dados, um dos focos da programação funcional. Podemos nos tentar
a criar operadores arbitrários, que podem ser usados para resolver problemas
específicos (por exemplo, um operador \texttt{.+} que faça soma de matrizes).
Porém, isso pode afetar bastante a legibilidade do código, gerando símbolos
imprevisíveis, específicos a uma pequena parte de uma única aplicação. Haskell
permite a criação de operadores arbitrários, o que, em muitos casos, produz
código extremamente difícil de se entender. Em Ipe, iremos evitar a criação de
operadores arbitrários, dando preferência a funções nomeadas. Os únicos operadores
serão definidos pela linguagem em si, e serão os mesmos em todos os programas Ipe.
Todos os operadores disponíveis em Ipe estão listados na \autoref{tab:operadores}.

\subsection{Ferramentas externas}

Em quase todos os exemplos que analisamos, foi necessário utilizar ferramentas
externas para auxiliar no desenvolvimento do programa. O principal exemplo são
as ferramentas de geração de código, presentes no ASP.NET, Phoenix e IHP. Enquanto
essas ferramentas podem ser extremamente úteis, o objetivo de Ipe é ser extremamente
simples de usar. Portanto, a única ferramenta que precisará ser instalada para
rodar programas Ipe será o compilador da linguagem.

Ao desenvolver as aplicações, também foi notável as ferramentas de auxílio em
desenvolvimento nos editores de código. Linguagens com tipagem dinâmica têm maior
dificuldade em sugerir compleções de código, e ferramentas como \textit{IntelliSense}.
A exceção é Javascript, pois os editores usam informações de tipo do \textit{Typescript}
por baixo do panos.

No futuro, o compilador de Ipe pode cumprir o papel de algumas dessas ferramentas,
como formatadores automáticos e \textit{linters}, além de disponibilizar dados
que podem ser usados por outras ferramentas, como editores de código.
