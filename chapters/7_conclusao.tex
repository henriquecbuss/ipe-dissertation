\phantomsection

\chapter{Conclusões finais}
\phantomsection

Este trabalho apresenta Ipe, uma linguagem de programação puramente funcional.
Ipe visa usar características de linguagens e \textit{frameworks} de desenvolvimento
de aplicações \textit{backend} modernos. Vimos algumas dessas características na
\autoref{subsec:js} e \autoref{subsec:python}, quando vimos aplicações \textit{backend}
em Javascript e Python, respectivamente. Também vimos o exemplo de uma aplicação
em Haskell, na \autoref{subsec:haskell}, e vimos que, embora o código seja bastante
resiliente, ele é extremamente complexo e difícil de compreender. Ipe tem o objetivo
de simplificar o uso da programação funcional, sem comprometer a expressividade
garantida por linguagens funcionais e diminuindo a barreira de entrada no mundo
funcional, seguindo o exemplo de Elm.

No \autoref{chapter:specification}, definimos as características da linguagem Ipe:
vimos definições de módulos, tipos e funções. Também discutimos sobre a arquitetura
e o \textit{runtime} Ipe no \autoref{chapter:programas-em-ipe}, que possibilitam
programas compostos apenas de funções puras, sem efeitos colaterais, mas que
sejam úteis no mundo real.

Nos próximos capítulos a serem escritos, iremos discutir sobre a implementação
de Ipe, usando a linguagem de programação Haskell, uma das principais linguagens
funcionais. Também iremos construir o \textit{runtime} Ipe, junto com alguns módulos
padrão que serão usados em todos os programas Ipe.

Até o momento, concluímos os objetivos específicos 1, 2 e 3 (definidos na
\autoref{sec:objectives}). Assim que tivermos a linguagem, o \textit{runtime} e
os módulos padrão em mãos, podemos construir a aplicação base descrita na
\autoref{sec:aplicacao-base} em Ipe, para que possamos comparar essa implementação
com as que fizemos no \autoref{chapter:trabalhos-relacionados} e verificar se
atingimos o restante dos objetivos.
