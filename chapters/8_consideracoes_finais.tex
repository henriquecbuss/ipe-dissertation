\phantomsection

\chapter{Considerações finais}
\phantomsection

Este trabalho apresenta Ipe, uma linguagem de programação puramente funcional.
Ipe visa usar características de linguagens e \textit{frameworks} de desenvolvimento
de aplicações \textit{backend} modernos. Vimos algumas dessas características na
\autoref{subsec:js} e \autoref{subsec:python}, quando vimos aplicações \textit{backend}
em Javascript e Python, respectivamente. Também vimos o exemplo de uma aplicação
em Haskell, na \autoref{subsec:haskell}, e vimos que, embora o código seja bastante
resiliente, ele é extremamente complexo e difícil de compreender. Ipe tem o objetivo
de simplificar o uso da programação funcional, sem comprometer a expressividade
garantida por linguagens funcionais e diminuindo a barreira de entrada no mundo
funcional, seguindo o exemplo de Elm.

No \autoref{chapter:specification}, definimos as características e a sintaxe da
linguagem Ipe: vimos definições de módulos, tipos e funções. Também discutimos
sobre a arquitetura e o \textit{runtime} Ipe no \autoref{chapter:programas-em-ipe},
que possibilitam programas compostos apenas de funções puras, sem efeitos colaterais,
mas que sejam úteis no mundo real.

No \autoref{chapter:resultados-obtidos}, colocamos em prática o que foi definido
nos capítulos anteriores, após implementar um compilador para Ipe, usando a linguagem Haskell.
Construímos a aplicação base, definida na \autoref{sec:aplicacao-base}, e vimos que, embora Ipe precise
de mais linhas de código, a implementação é bastante simples e fácil de compreender, além de ter
várias garantias, por conta de seu forte sistema de tipos e gerenciamento de sistemas colaterais.

Deste modo, concluímos todos os objetivos delineados na \autoref{sec:objectives}, e temos uma nova
linguagem puramente funcional, que serve para definir APIs REST, com fortes garantias de tipos.
Como trabalho futuro, podemos expandir os módulos padrão de Ipe, além de adicionar otimizações ao
compilador, dando um foco maior à performance da linguagem. Também poderíamos adicionar um sistema
que permita a interação entre Ipe e Javascript, para que programadores Ipe possam criar suas próprias
bibliotecas padrão.

Por fim, Ipe é uma linguagem de programação funcional, que busca ser simples e fácil de se usar,
ao mesmo tempo em que tenta capturar erros em tempo de compilação, diminuindo erros em tempo de execução.
Ipe é uma linguagem que busca ser útil no mundo real, e que pode ser usada para criar aplicações
\textit{backend} modernas, com fortes garantias de tipos e sem efeitos colaterais. Ipe é uma linguagem
que busca ser fácil de se aprender, por ter poucos conceitos e uma sintaxe simples. Desta forma, Ipe
apresenta vantagens sobre linguagens como Javascript e Python, que possuem tipagem fraca e dinâmica,
e também apresenta vantagens sobre linguagens como Haskell, que, embora possuam um sistema de tipos
forte e estático, possuem uma alta barreira de entrada. Ipe também precisa de menos código do que
\textit{frameworks} como ASP.NET, que se baseiam em ferramentas de geração automática de código, o
que pode facilitar a manutenção de aplicações Ipe. Por outro lado, Ipe ainda é uma linguagem nova,
e precisa de mais testes e mais aplicações reais para que possamos ter certeza de que ela cumpre
seus objetivos. Além disso, Ipe ainda precisa de mais ferramentas, como integração com editores de
código, bibliotecas de teste, e um gerenciador de pacotes, para que a comunidade possa compartilhar
código mais facilmente. Ipe também não apresenta otimizações de código, o que pode ser um problema
para aplicações \textit{backend}, que muitas vezes precisam de alta performance para atender várias
requisições simultaneamente.
