\phantomsection

\chapter{Cronograma}
\phantomsection

Até agora, já analisamos algumas outras linguagens, e colhemos inspirações para
o desenvolvimento de Ipe. Usamos estas informações no \autoref{chapter:specification}
para definir a linguagem Ipe. Agora, vamos nos concentrar na implementação da
linguagem de fato. Para tal, usaremos a linguagem de programação Haskell.  A
\autoref{tab:cronograma} descreve os próximos passos para o desenvolvimento de Ipe.

\begin{table}[htb]
    \caption{Cronograma de desenvolvimento para a linguagem Ipe}
    \label{tab:cronograma}
    \resizebox{\textwidth}{!}{%
        \begin{tabular}{p{9.5cm}p{4.65cm}}
            \toprule
            \textbf{Tarefa}                                             & \textbf{Tempo estimado para conclusão} \\
            \midrule
            \textit{Parsing} de código fonte                            & 2 semanas                              \\
            Checagem de tipos                                           & 1 semana                               \\
            Geração de código Javascript                                & 1 semana                               \\
            Construção do \textit{Runtime} e módulos padrão             & 3 semanas                              \\
            Construção da aplicação base em Ipe e análise de resultados & 2 semanas                              \\
            \bottomrule
        \end{tabular}
    }
\end{table}

Ao todo, estima-se que o desenvolvimento da linguagem Ipe levará cerca de mais 9
semanas, ou seja, 63 dias.
