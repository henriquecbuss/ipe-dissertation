\phantomsection

\chapter{Programas em Ipe}
\phantomsection

Para facilitar o desenvolvimento de aplicações Ipe, todos os programas contam com
bibliotecas padrão, que são apresentadas na \autoref{sec:libpadrao}. Ademais, para
manter a pureza dos programas Ipe, a linguagem disponibiliza um ambiente que
faz a gerência de efeitos colaterais, como acesso a disco ou rede. Esse ambiente
é apresentado na \autoref{sec:runtime}.

\section{Bibliotecas padrão}\label{sec:libpadrao}

Ipe oferece módulos para facilitar o uso de seus tipos primitivos (\texttt{Number}
e \texttt{String}), além de módulos para manipulação de listas e dicionários, e
representação de dados opcionais. Essas bibliotecas podem utilizar código Javascript
(a linguagem alvo da compilação) para utilizar otimizações e acessar dados que
não seriam possíveis de acessar em Ipe. Nem todas as bibliotecas serão apresentadas.

\subsection{Number}

Este módulo contém algumas funções para manipulação de números, como arrendondamento.
Além disso, como Ipe não possui operadores lógicos (em especial \texttt{==} e \texttt{!=}),
este módulo oferece funções e tipos para comparar números, como mostra o \autoref{lst:number-compare}.

\begin{lstlisting}[label={lst:number-compare},caption={Comparação de números em Ipe}]
type union CompareResult =
    | Smaller
    | Equal
    | Greater

compare : Number -> Number -> CompareResult
\end{lstlisting}

\subsection{String}

O módulo de \texttt{String} contém funções para manipulação de strings, como
concatenação, busca de substrings, e conversão para maiúsculas e minúsculas.

\subsection{Maybe e Result}

Como Ipe não possui nenhum tipo de exceção ou \texttt{null}, os módulos \texttt{Maybe}
e \texttt{Result} são utilizados para representar valores opcionais e resultados
de funções que podem falhar, respectivamente. O \autoref{lst:maybe-result-def}
mostra a definição destes tipos.

\begin{lstlisting}[label={lst:maybe-result-def},caption={Definição de \texttt{Maybe} e \texttt{Result}}]
type union Maybe content =
    | Nothing
    | Just content

type union Result error ok =
    | Error error
    | Ok ok
\end{lstlisting}


\subsection{List}

O módulo de \texttt{List} define uma das principais estruturas de dado de Ipe,
além de funções para manipular listas. Em Ipe, listas são representadas como
listas encadeadas, como mostra o \autoref{lst:list-def}

\begin{lstlisting}[label={lst:list-def},caption={Definição de listas em Ipe}]
type opaque List element =
    | Empty
    | Node element (List element)
\end{lstlisting}

\subsection{Dict}

Dicionários são coleções de chave-valor dinâmicos. O módulo \texttt{Dict} define
o tipo \texttt{Dict} e funções para manipular dicionários. Por questões de performance
e simplicidade, dicionários são implementados em Javascript.

\subsection{Subscription}

O tipo \texttt{Subscription} é usado pelo ambiente de execução (discutido na
\autoref{sec:runtime}) para sinalizar eventos que ocorrem no sistema e o programa
pode escolher ser notificado para tomar alguma ação.

\subsection{Task}

\textit{Tasks} (ou tarefas) são funções que solicitam ações no mundo exterior para
o ambiente de execução. Esse módulo disponibiliza funções para manipular tarefas,
como usar o resultado de uma como entrada de outra. O \autoref{lst:task-example}
mostra um exemplo de sequenciamento de tarefas.

\begin{lstlisting}[label={lst:task-example},caption={Exemplo de sequenciamento de tarefas}]
uploadFile : String -> String -> Task Error String
uploadFile =
    \fileId destinationUrl ->
        Db.getFilePathById fileId
        |> Task.andThen (\filePath -> File.read filePath)
        |> Task.andThen (\fileContent -> uploadFile fileContent destinationUrl)
\end{lstlisting}

A função acima recebe um identificador de arquivo e uma URL de destino, e lê
o caminho do arquivo no banco de dados, lê o arquivo, e envia o conteúdo para
a URL de destino. A função \texttt{Task.andThen} é usada para encadear as tarefas,
passando o resultado de uma para a outra. Este código é apenas um exemplo, pois
na realidade, teríamos que lidar com erros (usando o tipo \texttt{Result}), e
não apenas com o caso de sucesso.


\section{Runtime}\label{sec:runtime}

O \textit{Runtime} (ambiente de execução) é o responsável por realizar efeitos
colaterais, que podem ser executar \texttt{Tasks}, ou sinalizar o programa quando
uma \texttt{Subscription} for ativada. Por exemplo, o \textit{Runtime} é responsável
por ler o conteúdo de um arquivo e receber e enviar requisições HTTP. O \textit{Runtime}
é implementado em Javascript, e ele converte \texttt{Tasks} em código Javascript,
e retorna o resultado para o programa Ipe. Similarmente, quando um programa Ipe
adiciona uma nova \texttt{Subscription}, o \textit{Runtime} é responsável por
gerar Javascript que avise o programa Ipe quando necessário.

% TODO - Adicionar uma figura
