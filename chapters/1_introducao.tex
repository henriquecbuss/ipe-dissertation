\phantomsection

\chapter{Introdução}
\phantomsection

Nos últimos anos, os desenvolvedores de \textit{software} têm sido atraídos cada vez mais
ao paradigma de programação funcional. Em contraste com o paradigma de orientação
a objetos, o paradigma funcional visa simplificar o modelo mental dos programadores,
focando em funções puras, imutabilidade, e composição de funções (\textcite{functionalprogramming-future}).
Grandes linguagens de programação orientadas a objetos também estão indo na direção
da programação funcional, como Java, JavaScript, C\#, e Python. Embora esses
conceitos possam criar programas mais robustos e mais fáceis de testar, pode ser
difícil para desenvolvedores acostumados com outros paradigmas a aprenderem esse
novo modelo mental \cite{promisesoffp}.

O paradigma funcional não é algo concreto - é um conjunto de ideias que podem ser
adotadas em partes. Enquanto existem linguagens que implementam
algumas ideias da programação funcional (como as mencionadas acima), outras
tentam implementar o máximo possível - comumente chamadas de \textit{linguagens puras},
ou \textit{linguagens puramente funcionais}. Não existe uma definição exata para
o que é uma linguagem puramente funcional, mas elas geralmente têm alguns recursos,
como \textit{pattern matching}, funções puras (sem efeitos colaterais, como acesso
ao sistema de arquivos, ou mostrar uma mensagem na tela), tipagem forte e estática,
e inferência de tipos. Alguns exemplos de linguagens puramente funcionais são
\textit{Haskell} \cite{conceptionoffunctionalpl} e \textit{Elm} \cite{czaplicki2012elm}.

Linguagens de programação geralmente são criadas para um propósito específico. Por
exemplo, \textit{Javascript} foi criada para melhorar a experiência dos usuários
na \textit{web}. \textit{Elm} foi criada para simplificar a criação de aplicativos
\textit{web}, utilizando programação funcional para diminuir erros em tempo de
execução, e aumentar a produtividade dos desenvolvedores. Neste trabalho, vamos
explorar algumas das linguagens de programação utilizadas para criar aplicativos
\textit{backend} e, inspirados por \textit{Elm}, vamos criar
\textit{Ipe}\footnote{\textit{Elm} significa Olmo (uma árvore) em inglês. Ipê é uma árvore brasileira. Para simplificar o uso em inglês, vamos usar o nome \textit{Ipe}, sem acento.},
uma linguagem de programação puramente funcional, desenvolvida especificamente para
desenvolver aplicativos \textit{backend}, com o objetivo de simplificar o desenvolvimento
e diminuir a quantidade de erros em tempo de execução, além de diminuir a barreira
de entrada a linguagens puramente funcionais.

\section{Objetivos}\label{sec:objectives}

Com a devida contextualização, os objetivos deste trabalho são:

\subsection{Objetivo Geral}

Desenvolver Ipe, uma linguagem de programação puramente funcional, com foco em desenvolvimento de
aplicações \textit{backend} para a \textit{web}, e com objetivo de diminuir a barreira de entrada a
linguagens funcionais.

\subsection{Objetivos Específicos}

Para cumprir o objetivo geral, precisamos cumprir os seguintes objetivos específicos:

\begin{enumerate}
      \item Definir os requisitos para uma aplicação \textit{backend} com uma API
            REST, para servir de modelo.
      \item Implementar esses requisitos em algumas linguagens de programação,
            procurando por padrões, boas práticas, e dificuldades.
      \item Definir a linguagem \textit{Ipe}.
      \item Criar um compilador para \textit{Ipe}, que possibilite gerar código
            que possa ser executado.
      \item Implementar os requisitos definidos no primeiro item usando \textit{Ipe},
            para comparar com as implementações descritas no item 2.
      \item Analisar e discutir os resultados obtidos.
\end{enumerate}


\section{Escopo do Trabalho}

Por questões de limitação de tempo, a aplicação modelo será bem simples, e não
necessariamente representará uma aplicação real. Além disso, \textit{Ipe} é apenas
uma prova de conceito. Não é o objetivo deste trabalho criar uma linguagem de
programação completa, com otimizações, ferramentas, e bibliotecas extensas. O
objetivo é explorar as ideias da programação funcional no desenvolvimento
\textit{backend}. Por conta disso, a linguagem \textit{Ipe} não deve ser usada
no mundo real, em projetos reais. Dito isso, ainda queremos representar um caso
de uso comum para sistemas \textit{backend}, e é por isso que temos o objetivo
de construir uma API REST em Ipe.

O compilador de \textit{Ipe} deve funcionar em sistema operacional macOS, onde
vai ser desenvolvido, e pode não funcionar em outros sistemas.

\section{Método de Pesquisa}

Para cumprir os objetivos propostos, o desenvolvimento é dividido em 3 fases, que
são explicadas a seguir:

\subsection{Trabalhos Relacionados}

Para termos noção do que outras linguagens fazem para o desenvolvimento de
aplicações \textit{backend}, vamos explorar algumas linguagens de programação
ranqueadas como mais populares em índices como o índice \textcite{tiobeindex}, e analisar algumas
linguagens que inspiraram ou influenciaram no desenvolvimento de \textit{Ipe}.

\subsection{Implementação}

Nesta fase, vamos de fato implementar \textit{Ipe} - seu \textit{parser}, compilador
e tudo o que for necessário para executar programas escritos em \textit{Ipe}.
Vamos discutir decisões de sintaxe, funções padrões, estrutura de projetos.

Mostraremos como usar \textit{Ipe}, descrevendo estruturas de controle, sistema
de tipos, estruturas de dados, funções.

\subsection{Uso em Aplicações}

Vamos desenvolver a aplicação modelo utilizando \textit{Ipe}, além de outras
pequenas aplicações de exemplo. Vamos discutir as vantagens e desvantagens de
usar \textit{Ipe} para desenvolver aplicações \textit{backend}, e comparar os
resultados obtidos com \textit{Ipe} e os resultados obtidos com linguagens orientadas
a objetos, e também com outras linguagens funcionais.

\subsection{Avaliação dos resultados}

Após implementar a linguagem e a aplicação base, também conduziremos sessões
de entrevista onde coletaremos \textit{feedback} de outros desenvolvedores para
obter métricas de avaliação. Nessas sessões de entrevista, desenvolvedores serão
convidados a desenvolver a mesma aplicação base em Ipe. Ao final das entrevistas,
os convidados irão responder perguntas sobre vantagens e desvantagens de Ipe em
relação a outras linguagens, e poderemos ter métricas de comparação.
