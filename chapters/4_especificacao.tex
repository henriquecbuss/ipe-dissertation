\phantomsection

\chapter{Especificação da linguagem}\label{chapter:specification}
\phantomsection

Este capítulo tem o objetivo de definir as características da linguagem Ipe, como
sintaxe e semântica. As características serão demonstradas principalmente através
de código Ipe, para exemplificar o uso real, seguidos por sua definição sintática.
Para definir a sintaxe de Ipe, usaremos a notação EBNF (\textit{Extended Backus-Naur Form}),
como descrito em \cite{ebnfstandard}. Por simplicidade, algumas definições são
omitidas neste capítulo. Além disso, como iremos fazer o nosso próprio
\textit{parser}-- ao invés de usar um gerador automático --a gramática não será
otimizada, e alguns detalhes não serão apresentados, como espaços em branco
e comentários de bloco entre tokens. O \autoref{apendix:ebnf-syntax} contém a
definição sintática completa da linguagem.

\section{Comentários}

Ipe define três tipos de comentários, que são ignorados pelo compilador. O primeiro
tipo é o comentário de linha, que começa com \texttt{//}, e vai até o final da linha.
O segundo tipo é o comentário de bloco, que começa com \texttt{/*} e termina com
\texttt{*/}, podendo conter múltiplas linhas, ou terminar na mesma linha. O último
tipo de comentário se chama comentário de documentação, e é similar ao comentário
de bloco, mas começa com \texttt{/|*}, e deve ser colocado na linha acima de uma
definição, ou no começo do arquivo, logo após a declaração de módulo. Comentários
de documentação têm a finalidade de documentar código, e serão usados principalmente por
futuras ferramentas de análise de código. Por exemplo, ao manter o cursor do
mouse sobre uma função em um editor de texto, o editor pode mostrar o comentário
de documentação. No \autoref{lst:comentarios} são demonstrados alguns exemplos
de comentários. No \autoref{grammar:comentarios} são apresentados os símbolos
não terminais que definem comentários.

\begin{lstlisting}[label={lst:comentarios},caption={Exemplos de comentários}]
// Comentário de linha

/* Comentário de bloco
   Que pode ter várias linhas
*/

/|* Comentário de documentação
    Aqui, podemos descrever uma função ou um módulo 
*/
\end{lstlisting}

\begin{lstlisting}[label={grammar:comentarios},caption={Definição de comentários em EBNF},escapechar=`,numbers=none]
`\begin{bnfgrammar}
    line comment ::= "\slash\slash", \{? qualquer caractere ?\} - "\textbackslash n", "\textbackslash n";
    ;;
    block comment ::= "\slash*", \{? qualquer caractere ?\} - "*\slash", "*\slash";
    ;;
    doc comment ::= "\slash*$\vert$", \{? qualquer caractere ?\} - "*\slash", "*\slash";
\end{bnfgrammar}`
\end{lstlisting}


\section{Módulos}

Programas escritos em Ipe são organizados em módulos. Cada módulo representa um
arquivo, que começa com letra maiúscula e termina com a extensão \texttt{.ipe}
(por exemplo, \texttt{Main.ipe}). Dentro de cada módulo, podem existir várias
definições (mas ao menos uma), e cada módulo pode exportar várias funções (cada módulo
deve exportar ao menos uma definição). Falaremos mais sobre definições na
\autoref{sec:definicoes}. Para declarar um módulo, a primeira linha do arquivo deve
ser no formato \texttt{module NomeDoModulo exports <lista de definições exportadas>}.
Módulos dentro de pastas devem conter os nomes das pastas, separados por pontos.
Por exemplo, o arquivo \texttt{Pasta1/Pasta2/Modulo.ipe} deve conter a linha
\texttt{module Pasta1.Pasta2.Modulo exports <lista de definições exportadas>}.
A seguir, demonstramos alguns tipos de definição de módulo.

\begin{lstlisting}[label={lst:modulos},caption={Exemplo de módulos}]
module Main exports [ main ]
\end{lstlisting}

No \autoref{lst:modulos}, o módulo \texttt{Main} reside no arquivo \texttt{Main.ipe},
e exporta a definição chamada \texttt{main}.

\begin{lstlisting}[label={lst:modulos2},caption={Exemplo de módulo em subpasta que exporta várias definições}]
module Pasta1.Pasta2.Modulo exports
    [ funcaoA
    , tipoA
    , funcaoB
    , funcaoC
    ]
\end{lstlisting}

No \autoref{lst:modulos2}, o módulo \texttt{Pasta1.Pasta2.Modulo} reside no arquivo
\texttt{Modulo.ipe}, que está dentro da pasta \texttt{Pasta2}, que está dentro da
pasta \texttt{Pasta1}. O módulo exporta as definições \texttt{funcaoA}, \texttt{funcaoB},
\texttt{funcaoC} e \texttt{tipoA}. A lista de exportações pode percorrer múltiplas
linhas, e a ordem em que as definições aparecem não importa.

\begin{lstlisting}[label={lst:modulos3},caption={Exemplo de módulo com comentário de documentação}]
module ModuloComComentario exports [ funcaoA ]

/|* Comentário de documentação
    Aqui podemos descrever o módulo e suas definições
*/
\end{lstlisting}

No \autoref{lst:modulos3}, o módulo \texttt{ModuloComComentario} possui um comentário
de documentação, que pode ser usado por ferramentas para melhorar a exploração
do código.

A vantagem de organizar o código em módulos é que módulos podem importar outros
módulos, através da palavra-chave \texttt{import}. Isso é demonstrado no
\autoref{lst:modulos-import}.

\begin{lstlisting}[label={lst:modulos-import},caption={Exemplo de importação de módulos}]
module Main exports [ main ]

/|* Os comentários de documentação vêm antes das importações
*/

import ModuloA
import Pasta1.Pasta2.ModuloB as ModuloB
\end{lstlisting}

No \autoref{lst:modulos-import}, também podemos ver que é possível importar um
módulo com um apelido, usando a palavra-chave \texttt{as}. Isso é útil quando
um módulo possui um nome muito longo, ou quando outro nome faz mais sentido no
contexto do módulo atual. Quando um módulo é importado com apelido, seu nome
original não pode mais ser usado para se referir às definições daquele módulo.

O \autoref{grammar:modulos} mostra a definição de módulos em EBNF.

\begin{lstlisting}[label={grammar:modulos},caption={Definição de módulos em EBNF},escapechar=`,numbers=none]
`\begin{bnfgrammar}
    upper identifier ::= uppercase letter, \{letter $\vert$ digit $\vert$ "\_" \};
    ;;
    module name ::= \{upper identifier, "."\}, upper identifier;
    ;;
    exported item ::= letter, \{letter $\vert$ digit $\vert$ "\_"\};
    ;;
    export list ::= "[", \{exported item\}-, "]";
    ;;
    module decl ::= "module", module name, "exports", export list;
    ;;
    import decl ::= "import", module name, [ "as", module name ];
    ;;
    module ::= module decl, [doc comment], \{import decl\};
\end{bnfgrammar}`
\end{lstlisting}

\section{Definições}\label{sec:definicoes}

Nesta seção, discutiremos os dois tipos de definições em Ipe: tipos e funções.
Definições são itens definidos a nível de módulo, ou seja, podem ser exportadas
para serem usadas em outros módulos.

\subsection{Tipos}\label{sec:tipos}

Ipe possui um sistema de tipos muito parecido com o de Elm, e tem o objetivo de
ser o mais simples possível, sem abandonar expressividade. Por isso, existem
apenas três tipos primitivos: \texttt{Number}, \texttt{String} e \texttt{Never},
que representam números (inteiros e reais), cadeias de caracteres, e valores que
nunca podem ser construídos respectivamente. Números podem ter um ponto,
para separar as casas decimais. \texttt{String}s são definidas entre aspas simples.

Além disso, existe o tipo \texttt{Record}, que representa uma coleção de pares
chave-valor pré-definidos. \texttt{Records} são construídos entre chaves, e cada
par chave-valor é separado por vírgula. As chaves e seus valores são separados por
dois pontos. Para acessar um elemento de um \texttt{Record}, basta usar a sintaxe
\texttt{record.chave}.


Em Ipe, todos os valores e todas as definições são tipados (possuem um tipo). Para
demonstrar os tipos dos valores, usamos o caractere \texttt{:}, seguido pelo tipo.
O \autoref{lst:anotacoes} demonstra a anotação de tipos, enquanto o
\autoref{grammar:primitives} mostra as regras em EBNF para declaração dos tipos
primitivos (\texttt{Number}, \texttt{String} e \texttt{Record}).

\begin{lstlisting}[label={lst:anotacoes},caption={Exemplo de anotação de tipos}]
5 : Number
3.14 : Number
'abc' : String
{ a: 1, b: 'abc' } : { a : Number, b : String }
\end{lstlisting}

\begin{lstlisting}[label={grammar:primitives},caption={Definição dos tipos primitivos em Ipe em EBNF. A regra \texttt{expression} está descrita no \autoref{apendix:ebnf-syntax}},escapechar=`,numbers=none]
`\begin{bnfgrammar}
    number ::= ["$-$"], \{digit\}-, [".", \{digit\}-];
    ;;
    string ::= "'", \{? qualquer caractere ?\} - ("\textbackslash n" $\vert$ "'"), "'";
    ;;
    field decl ::= lower identifier, "$\colon$", expression;
    ;;
    record ::= "\{", [field decl, \{",", field decl\}], "\}";
\end{bnfgrammar}`
\end{lstlisting}

A verdadeira força do sistema de tipos de Ipe está na definição de novos tipos.
Ipe permite três modalidades de novos tipos:

\begin{itemize}
    \item Apelidos: são usados para dar um novo nome a algum outro tipo qualquer.
          São úteis para abreviar tipos compridos, como \texttt{Record}s com vários
          campos. Para definir um novo apelido, usamos \texttt{type alias}, como
          no \autoref{lst:type-alias}.
    \item Uniões: são usados para definir novos tipos, e são definidos com \texttt{type union}.
          Discutiremos mais sobre uniões no \autoref{lst:type-unions}.
    \item Opacos: são iguais às uniões, mas suas variantes não são exportadas para
          outros módulos, ou seja, tipos opacos só podem ser construídos e manipulados
          no módulo onde foram definidos. Tipos opacos são úteis para definir uma interface
          específica ao redor de tipos, pois seus módulos podem exportar funções
          que manipulam os tipos de forma específica, e é impossível que eles sejam manipulados
          de qualquer outra maneira. São definidos com \texttt{type opaque}, como
          mostra o \autoref{lst:type-opaque}.
\end{itemize}


\begin{lstlisting}[label={lst:type-alias},caption={Exemplo de apelido de tipo}]
type alias Pessoa = { nome : String, idade : Number }
\end{lstlisting}

Um apelido de tipo serve simplesmente para abreviar um tipo. Outras definições
poderiam se referir ao tipo \texttt{\{ nome: String, idade : Number \}} simplesmente
como \texttt{Pessoa}.

\begin{lstlisting}[label={lst:type-unions},caption={Exemplo de união de tipos}]
type union Permissao =
    | ConvidarNovosUsuarios
    | RemoverUsuarios

type union Usuario =
    | Admin { nome : String, permissoes : List Permissao }
    | UsuarioCadastrado { nome : String, email : String }
    | UsuarioAnonimo
\end{lstlisting}

Podemos ver que as uniões de tipos são definidas por uma coleção de uma ou mais
etiquetas (\textit{tags}, em inglês), e cada etiqueta pode ter um \texttt{Record}
como argumento. Veremos mais como utilizar uniões de tipos na \autoref{sec:pattern-matching}.

\begin{lstlisting}[label={lst:type-opaque},caption={Exemplo de tipos opacos}]
type opaque Permissao =
    | ConvidarNovosUsuarios
    | RemoverUsuarios

type opaque Usuario =
    | Admin { nome : String, permissoes : List Permissao }
    | UsuarioCadastrado { nome : String, email : String }
    | UsuarioAnonimo
\end{lstlisting}

Tipos opacos são definidos e funcionam exatamente como uniões de tipos, mas suas
etiquetas não são exportadas para outros módulos.

O nome de um tipo é sempre escrito com a primeira letra maiúscula, para facilitar
a distinção entre tipos e funções.

\subsubsection{Parâmetros de tipos}

Ipe permite definir tipos que recebem outros tipos como parâmetro. É assim que
estruturas como \texttt{Lists} e \texttt{Dicts} funcionam -- elas recebem parâmetros
de tipo para que possam ser genéricas. Todos os tipos mencionados acima (apelidos,
uniões e tipos opacos) podem receber parâmetros de tipo. Os nomes dos parâmetros
de tipos podem ser qualquer identificador que comece com uma letra minúscula. O
\autoref{lst:parametros-tipos} mostra como definir um tipo com parâmetro.

\begin{lstlisting}[label={lst:parametros-tipos},caption={Exemplo de como definir um tipo com parâmetro}]
type opaque Pilha elemento =
    | Pilha { itens : List elemento }

type alias PilhaDeNumeros = Pilha Number
\end{lstlisting}

O tipo \texttt{Pilha} recebe um parâmetro de tipo chamado \texttt{elemento}. Dentro
da definição de \texttt{Pilha}, podemos utilizar \texttt{elemento} para nos referirmos
ao tipo que foi passado como parâmetro. Com esse mecanismo, podemos definir tipos
genéricos. O tipo \texttt{PilhaDeNumeros} representa uma pilha especificamente de
números. O \autoref{grammar:custom-types} mostra a gramática para a definição de
apelidos, uniões e tipos opacos.

\begin{lstlisting}[label={grammar:custom-types},caption={Definição de novos tipos em EBNF},escapechar=`,numbers=none]
`\begin{bnfgrammar}
    custom type ::= \{upper identifier, "."\}, upper identifier, \{any type\};
    ;;
    any type ::= custom type
               | lower identifier
               | record annotation;
    ;;
    record ann field ::= lower identifier, "$\colon$", any type;
    ;;
    record annotation ::= "\{", [record ann field, \{",", record ann field\}] "\}";
    ;;
    alias ::= upper identifier, parameters, "=", any type;
    ;;
    union ::= upper identifier, parameters, "=", \{constructor\}-;
    ;;
    constructor ::= "$\vert$", upper identifier, [record annotation];
    ;;
    new type ::= "type alias", alias
        | "type union", union
        | "type opaque", union;
\end{bnfgrammar}`
\end{lstlisting}

\subsection{Funções}

Enquanto tipos são usados para representar modelos e domínios, funções são
usadas para manipular esses modelos e domínios. Juntos, funções e tipos são a
base da programação funcional. Diferentemente de outras linguagens, Ipe possui
uma única forma de definir funções, como mostra o \autoref{lst:funcoes}.

\begin{lstlisting}[label={lst:funcoes},caption={Exemplo de como definir uma função}]
\argumento1 argumento2 ->
    soma = argumento1 + argumento2;
    dobro = soma * 2;
    dobro * 4
\end{lstlisting}

Na linha 1, definimos os argumentos da função (\texttt{argumento1} e \texttt{argumento2}).
Argumentos podem receber qualquer identificador que comece com uma letra minúscula.
Nas linhas 2 e 3, definimos duas variáveis locais (\texttt{soma} e \texttt{dobro}).
Podemos incluir quantas declarações quisermos no corpo da função, mas as variáveis
declaradas devem ter nomes únicos (não podem repetir o nome de outra variável). A
expressão da última linha define o retorno da função (\texttt{dobro * 4}). Todas as linhas do corpo
de uma função, com exceção da última, devem ser terminadas em ponto e vírgula (;).

Esta sintaxe de definição de funções serve tanto para funções anônimas quanto para
funções nomeadas. Para definir uma função nomeada, basta dar um nome à função, como
mostra o \autoref{lst:funcoes-nomeadas}.

\begin{lstlisting}[label={lst:funcoes-nomeadas},caption={Exemplo de como definir uma função nomeada}]
somaEMultiplicaPor8 : Number -> Number -> Number
somaEMultiplicaPor8 =
    \argumento1 argumento2 ->
        soma = argumento1 + argumento2;
        dobro = soma * 2;
        dobro * 4
\end{lstlisting}

O \autoref{lst:funcoes-nomeadas} também mostra como definir o tipo de uma função.
Basta repetir o nome da função, seguido dos tipos dos argumentos e do tipo de retorno,
separados por \texttt{->}. O último tipo é o tipo de retorno.

Incluir assinaturas de funções é opcional, pois Ipe consegue inferir os tipos de
qualquer programa válido. Porém, adicionar as assinaturas de tipo pode ajudar a
produzir melhores erros de compilação, e também pode ajudar a documentar o código.

O último tipo de definição é o de constantes, que, na verdade, são apenas funções
que recebem zero argumentos. O \autoref{lst:constantes} mostra como definir uma
constante.

\begin{lstlisting}[label={lst:constantes},caption={Exemplo de como definir uma constante}]
resposta : Number
resposta = 42
\end{lstlisting}

Para executar uma função, passamos os argumentos, separados por espaço (sem parênteses
e sem vírgulas), como mostra o \autoref{lst:chamando-funcoes}.

\begin{lstlisting}[label={lst:chamando-funcoes},caption={Exemplo de como chamar uma função}]
numero96 : Number
numero96 = somaEMultiplicaPor8 4 8
\end{lstlisting}

Podemos usar parênteses para definir expressões maiores, como mostra o
\autoref{lst:chamando-funcoes-com-expressoes}.

\begin{lstlisting}[label={lst:chamando-funcoes-com-expressoes},caption={Exemplo de como chamar uma função com uma expressões maiores}]
numero96 : Number
numero96 = somaEMultiplicaPor8 (2 + 2) ((1 + 1) * 4)
\end{lstlisting}

Também é possível passar funções como argumento, como demonstrado no \autoref{lst:chamando-funcoes-higher-order}.

\begin{lstlisting}[label={lst:chamando-funcoes-higher-order},caption={Exemplo de como passar uma função como argumento}]
type List element =
    | Empty
    | Node { head : element, tail : List element }

map : (element -> mappedElement) -> List element -> List mappedElement
map =
    \mapFn list ->
        match list with
            | Empty -> Empty
            | Node node ->
                Node { head = mapFn node.head, tail = map mapFn node.tail }

soma2 : Number -> Number
soma2 =
    \x -> x + 2

soma2EmLista : List Number -> List Number
soma2EmLista =
    \lista -> map soma2 lista

soma2EmListaAnonimo : List Number -> List Number
soma2EmListaAnonimo =
    \lista -> map (\x -> x + 2) lista
\end{lstlisting}

O \autoref{grammar:functions} mostra a definição sintática de funções em Ipe.
Podemos ver que funções são apenas estruturas que recebem argumentos, e retornam
uma expressão.

\begin{lstlisting}[label={grammar:functions},caption={Definição de funções em EBNF, onde \texttt{tl} significa \textit{top level}},escapechar=`,numbers=none]
`\begin{bnfgrammar}
    fn args ::= \{lowercase identifier\};
    ;;
    function ::= "\textbackslash", fn args, "$-$>", \{attribution\}, expression;
    ;;
    tl annotation ::= lowercase identifier, "$\colon$", \{any type, "$-$>"\}, any type;
    ;;
    attribution ::= lowercase identifier, "$=$", expression, "$;$";
    ;;
    tl left ::= [doc comment], [tl annotation], lowercase identifier;
    ;;
    tl value ::= tl left, "$=$", expression;
\end{bnfgrammar}`
\end{lstlisting}

\section{Pattern matching}\label{sec:pattern-matching}

\textit{Pattern matching} é o que possibilita que usemos todos os poderes dos
tipos que definimos na \autoref{sec:tipos}. Com \textit{pattern matching}, podemos
extrair dados de tipos complexos. Diferente da maioria das linguagens, Ipe não
possui instruções \texttt{if}. Em vez disso, podemos usar \textit{pattern matching}.
O \autoref{lst:pattern-match-if} mostra como podemos fazer o mesmo que uma
instrução \texttt{if} usando a instrução \texttt{match ... with}.


\begin{lstlisting}[label={lst:pattern-match-if},caption={Substituto de if usando \textit{pattern matching}}]
type union Bool =
    | True
    | False

greet : Bool -> String
greet =
    \isHuman ->
        match isHuman with
            | True -> "Hello, human!"
            | False -> "Beep boop"
\end{lstlisting}

O \autoref{lst:pattern-match-correct} mostra uma versão mais idiomática,
que usa um tipo específico para o contexto da aplicação.

\begin{lstlisting}[label={lst:pattern-match-correct},caption={Maneira idiomática da função greet}]
type union Being =
    | Human { name : String }
    | Robot

greet : Being -> String
greet =
    \being ->
        match being with
            | Human human -> String.concat [ "Hello, ", human.name , "!" ]
            | Robot -> "Beep boop"
\end{lstlisting}

Podemos usar a instrução \texttt{match ... with} com qualquer tipo de dado! Por
exemplo, podemos usar \textit{pattern matching} com \texttt{Numbers}, como mostra
o \autoref{lst:pattern-match-number}.

\begin{lstlisting}[label={lst:pattern-match-number},caption={Pattern matching com números}]
fibonacci : Number -> Number
fibonacci =
    \n ->
        match n with
            | 0 -> 0
            | 1 -> 1
            | _ -> fibonacci (n - 1) + fibonacci (n - 2)
\end{lstlisting}

\textit{Pattern matching} em Ipe é exaustivo, ou seja, todas as variações de um
tipo devem ser tratadas. No \autoref{lst:pattern-match-number}, precisaríamos
tratar todos os números possíveis. A primeira linha (de cima para baixo) que
resultar em um \textit{match} é a linha executada. Ipe também possibilita o uso
de \texttt{\_} para fazer \textit{match} em qualquer valor de um tipo, o que é
útil para tipos com infinitas etiquetas, como \texttt{Number} e \texttt{String},
mas também pode ser usado em tipos com etiquetas finitas. O \autoref{grammar:pattern-matching}
mostra a definição sintática de \textit{pattern matching} em Ipe.

\begin{lstlisting}[label={grammar:pattern-matching},caption={Definição de \textit{pattern matching} em EBNF},escapechar=`,numbers=none]
`\begin{bnfgrammar}
    type destruct ::= \{upper identifier, "."\}, upper identifier, \{pattern\};
    ;;
    pattern ::= "\_"
        | number
        | string
        | type destruct;
    ;;
    match case ::= "$\vert$", pattern, "$-$>", \{attribution\}, expression;
    ;;
    match ::= "match", expression, "with", \{match case\}-;
\end{bnfgrammar}`
\end{lstlisting}

\section{Operadores}

A \autoref{tab:operadores} apresenta todos os operadores disponíveis na
linguagem Ipe. Embora os operadores tenham uma assinatura de tipo como qualquer
outra função, eles são usados de maneira infixa. Ou seja, o operador \texttt{+}
possui a assinatura \texttt{Number -> Number -> Number}, mas o primeiro argumento
fica à esquerda do símbolo \texttt{+}. Poderíamos definir uma função
\texttt{soma} com a mesma assinatura, e usar como \texttt{soma 1 2}. Com os
operadores, o primeiro argumento vai à esquerda do operador, e o segundo à direita,
como em \texttt{1 + 2}.

\begin{table}[htb]
    \caption[Operadores em Ipe]{Operadores em Ipe}
    \label{tab:operadores}
    \resizebox{\textwidth}{!}{%
        \begin{tabular}{p{2.03cm}p{6.10cm}p{6.10cm}}
            \toprule
            \textbf{Operador} & \textbf{Assinatura}                 & \textbf{Descrição}                 \\
            \midrule
            +                 & \texttt{Number -> Number -> Number} & Soma dois números                  \\
            -                 & \texttt{Number -> Number -> Number} & Subtrai dois números               \\
            /                 & \texttt{Number -> Number -> Number} & Divide dois números                \\
            *                 & \texttt{Number -> Number -> Number} & Multiplica dois números            \\
            \textasciicircum  & \texttt{Number -> Number -> Number} & Eleva o primeiro número ao segundo \\
            |>                & \texttt{a -> (a -> b) -> b}         & Aplica um argumento a uma função   \\
            <|                & \texttt{(a -> b) -> a -> b}         & Aplica uma função a um argumento   \\
            \bottomrule
        \end{tabular}
    }
\end{table}

Podemos ver que Ipe possui um baixo número de operadores. Isso é proposital, pois
queremos que a linguagem seja o mais simples possível. Isso também é uma consequência
de não termos o tipo booleano na linguagem.

Da tabela, talvez os operadores com maior destaque são |> e <|, pois estes operadores
ajudam a criar séries de transformações, que são muito comuns em programação funcional.
O \autoref{lst:pipes} ajuda a ilustrar o uso deles.

\begin{lstlisting}[label={lst:pipes},caption={Exemplos de uso dos operadores de pipe}]
comPipeDireito : Number -> Number -> Number -> Number
comPipeDireito =
    \n1 n2 n3 -> operacao3 n3 |> operacao2 n2 |> operacao1 n1

// é equivalente a

comPipeEsquerdo : Number -> Number -> Number -> Number
comPipeEsquerdo =
    \n1 n2 n3 -> operacao1 n1 <| operacao2 n2 <| operacao3 n3

// é equivalente a

semPipe : Number -> Number -> Number -> Number
semPipe =
    \n1 n2 n3 -> operacao1 n1 (operacao2 n2 (operacao3 n3))
\end{lstlisting}

O \autoref{grammar:expression} mostra a definição de expressões em Ipe, que combinam
operadores e chamadas de função para construir novos valores. Para facilitar a
compreensão da gramática, ela é apresentada com recursão à esquerda.

\begin{lstlisting}[label={grammar:expression},caption={Definição de expressões em EBNF},escapechar=`,numbers=none]
`\begin{bnfgrammar}
    expression ::= expression, "$\vert$>", expression
                 | expression, "$<$$\vert$", expression
                 | function
                 | match
                 | exponentiation;
    ;;
    exponentiation ::= exponentiation, "\textasciicircum", term
                     | term;
    ;;
    term ::= term, ("+" $\vert$ "$-$"), factor
           | factor;
    ;;
    factor ::= factor, ("\slash" $\vert$ "*"), primary
             | primary;
    ;;
    primary ::= number
              | string
              | record
              | list
              | variable name, \{primary\}
              | "(", expression, ")";
    ;;
    list ::= "[", [expression, \{",", expression\}], "]";
    ;;
    variable name ::= \{uppercase identifier, "."\}, record access;
    ;;
    record access ::= lowercase identifier, ".", record access
                    | lowercase identifier;
\end{bnfgrammar}`
\end{lstlisting}
