
\phantomsection

\chapter{Construção do Compilador}\label{chapter:construcao-do-compilador}
\phantomsection

O compilador de Ipe, escrito em Haskell, é dividido em 4 partes: análise sintática e léxica,
transformação, checagem de tipos, e emissão de código. Discutiremos cada uma dessas partes nas seções
seguintes.  Finalmente, na \autoref{sec:processo_de_compilacao}, falaremos sobre o processo de
compilação como um todo: como os arquivos são escaneados, e como as diferentes partes do processo se
comunicam.

\section{Análise sintática e léxica}

A etapa de análise sintática e léxica é a primeira etapa do compilador Ipe. Esta etapa também é
conhecida como a etapa de \textit{parsing} \cite{dragonbook}, e ela é responsável por transformar texto que representa
um programa Ipe em um formato de árvore, que representa a gramática de Ipe. Durante esta etapa, o
compilador verifica se todos os símbolos de entrada são válidos, ao mesmo tempo em que verifica se a
ordem dos símbolos também é válida. Caso haja algum erro sintático ou léxico no texto analisado,
o processo de compilação para, e o erro é reportado ao usuário.


\section{Transformação de código}

A etapa de transformação de código é responsável por passar pela árvore gerada na etapa de
\textit{parsing} e normalizar o código e sua estrutura. Na versão inicial de Ipe, essa etapa só é
responsável por adicionar os construtores de uniões de tipos na lista de exportação de cada módulo,
mas outras transformações poderiam ser adicionadas no futuro, como algumas otimizações, substituições
de variáveis por constantes, ou qualquer simplificação que torne o trabalho das próximas etapas mais
fácil (desde que o funcionamento do código gerado não seja afetado). Como essa etapa apenas transforma
um programa válido em outro programa válido, ela não gera erros.

\section{Checagem de tipos}

A checagem de tipos é a etapa que garante que um programa Ipe é seguro de se executar. Esta etapa é
responsável por assegurar que todas as funções invocadas recebam o número de argumentos corretos, e
que estes argumentos sejam do tipo correto, além de verificar a exaustividade de \textit{pattern matching}
e a unicidade de \textit{pattern matching} em valores. O sistema de tipos de Ipe é baseado no sistema
de tipos de Hindley-Milner \cite{principaltypeschemes}, e usa uma adaptação do algoritmo W \cite{understaingalgorithmw},
com adição de uniões de tipos, tipos opacos, \textit{records} e listas.

O algoritmo de checagem de tipos se baseia no conceito de unificação de tipos. A unificação de tipos
é o processo de encontrar uma substituição para variáveis de tipos que torna dois tipos iguais. Por
exemplo, o tipo \texttt{a -> b} pode ser unificado com o tipo \texttt{Number -> String} se a variável
\texttt{a} for substituída por \texttt{Number}, e a variável \texttt{b} for substituída por
\texttt{String}, assim como o tipo \texttt{Number} pode ser unificado com o tipo \texttt{Number}.
No caso de dois tipos não poderem ser unificados (por exemplo, se tentarmos unificar \texttt{Number}
e \texttt{String}), o compilador deve reportar um erro e abortar o processo de compilação. O algoritmo
original é estendido, para termos suporte a uniões de tipos, tipos opacos, \textit{records} e listas,
além de outros tipos de expressão, como definição de funções, \textit{pattern matching}, e outros
tipos de operadores binários, como os operadores de \textit{pipe}. Seguindo este processo, o
compilador Ipe é capaz de inferir os tipos de qualquer programa Ipe válido, de modo que anotações de
tipo são opcionais. Porém, é recomendável que anotações de tipo sejam usadas, para gerar melhores
mensagens de erro, e para melhor documentação do programa. Caso esta etapa seja bem sucedida, nenhuma
saída é gerada, e o processo de compilação continua.


\section{Emissão de código}

Por fim, depois de analisar todo o código, a estrutura em árvore que representa o programa em Ipe é
transformada em uma estrutura em árvore cujos nós são compostos de texto, que representa um programa
em Javascript, e esta árvore é transformada em texto. Ou seja, cada nó da árvore que representa o
programa em Ipe é transformado em um texto que representa uma parte do programa em Javascript. Após
converter toda a árvore, os nós são juntados, de forma que gere um programa Javascript válido.

Cada módulo Ipe gera um arquivo Javascript com o mesmo nome, e com a extensão
\texttt{.ipe.js}. Por exemplo, o módulo \texttt{Root.ipe} gera o arquivo \texttt{Root.ipe.js}. Como
sabemos das etapas anteriores que o código é válido, não existe a possibilidade de gerar erros nesta
etapa.

\section{Processo de compilação}\label{sec:processo_de_compilacao}

O processo de compilação é iniciado chamando um aplicativo de linha de comando (\textit{CLI}), onde
o usuário deve indicar o arquivo de entrada do programa, que deve expor uma função chamada
\texttt{main}. O \autoref{code:cli} mostra  um exemplo de chamada do compilador, passando
\texttt{Root.ipe} como arquivo de entrada, e especificando o diretório \texttt{output} como diretório
de saída (diretório onde os arquivos Javascript deverão ser gerados).

\begin{lstlisting}[language=Bash,label={code:cli},caption={Chamada do aplicativo de linha de comando}]
$ ipe build Root.ipe -o output
\end{lstlisting}

Ao executar o comando acima, o compilador irá fazer o \textit{parsing} do módulo \texttt{Root.ipe},
utilizando a biblioteca Haskell MegaParsec, e de todos os módulos que ele importa, e aplicar a etapa
de transformação de código em cada um, a fim de normalizar o código, gerando uma árvore de sintaxe
abstrata para cada módulo, que facilita a aplicação das próximas etapas de compilação. Em seguida,
o compilador começa a aplicar a etapa de checagem de tipos, utilizando as árvores de sintaxe abstrata
geradas anteriormente, aplicando um algoritmo inspirado no Algoritmo W \cite{understaingalgorithmw}.
A checagem de tipos começa pelo módulo de entrada (no exemplo, \texttt{Root.ipe}). Para cada módulo
importado pelo arquivo sendo analisado, o compilador verifica todos os módulos que estão sendo importados:

\begin{itemize}
      \item Se o módulo importado já foi analisado, o compilador continua a checagem de tipos do módulo
            atual.
      \item Se o módulo importado ainda não foi analisado, o compilador analisa o módulo importado, e
            depois continua com a checagem de tipos do módulo atual.
\end{itemize}

Os resultados da checagem de tipos são armazenados em uma tabela de símbolos. Ao terminar a checagem
de tipos de um módulo, apenas suas definições exportadas são mantidas na tabela de símbolos global.
Ao terminar a checagem de tipos de um módulo, o compilador transforma cada nó da árvore em uma
representação textual, que corresponde a trechos de código Javascript. A árvore de texto Javascript
então é reduzida, de seus galhos à sua raiz, juntando os nós em um programa completo em Javascript.
Depois de checar os tipos de todos os módulos importados (direta ou indiretamente) pelo módulo de
entrada, o compilador gera a estrutura de pastas e escreve os arquivos com o conteúdo em Javascript
de cada módulo. Deste modo, apenas módulos que aparecem na cadeia de dependências do módulo de entrada
são analisados e transformados em Javascript. Módulos que não são usados não são analisados, e
portanto não são transformados em Javascript.

Por fim, o compilador baixa as bibliotecas padrão (descritas na \autoref{sec:libpadrao}), definidas
em Javascript, para que o programa possa ser executado. Para executar o programa, basta usar algum
\textit{runtime} Javascript, como Node, Deno ou Bun. O código Javascript de algumas bibliotecas
padrão utiliza funções oferecidas pelo Bun, portanto é recomendável utilizar Bun para executar
programas Ipe.
