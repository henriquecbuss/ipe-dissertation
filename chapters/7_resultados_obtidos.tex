\phantomsection

\chapter{Resultados obtidos}\label{chapter:resultados-obtidos}
\phantomsection

Neste capítulo, será apresentada a implementação em Ipe do sistema descrito na
\autoref{sec:aplicacao-base}. Depois, iremos comparar alguns detalhes da implementação com as
implementações em outras linguagens, a fim de mostrar as vantagens e desvantagens de cada uma.

\section{Implementação da Aplicação Base em Ipe}

A implementação em Ipe usa várias das bibliotecas padrão de Ipe, apresentadas na \autoref{sec:libpadrao},
com maior foco nas bibliotecas de \texttt{JSON} e \texttt{Http}, o que é comum em \textit{API}s
\textit{REST}, por ser o principal meio de comunicação com o mundo exterior. Esta seção será dividida
em partes menores, cada uma descrevendo uma parte da implementação. Por simplicidade, todo o código
que será apresentado está em um único arquivo, \texttt{Root.ipe}. Para aplicações maiores, recomenda-se
que o código seja dividido em vários arquivos diferentes. Durante a apresentação do código, partes
que poderiam ser definidas em outros arquivos serão destacadas. Nos códigos apresentados, não serão
mostradas as importações de módulos, que são feitas no início do arquivo, e são necessárias para
que as funções e tipos dos módulos importados possam ser usados.

\subsection{Descrição do contexto}

Como a maior parte dos programas em Ipe, começamos definindo um tipo para o nosso modelo (ou contexto,
quando usando a biblioteca \texttt{Http}). Para a aplicação base, o contexto deve saber o último
identificador usado para um usuário, e um dicionário de usuários, como mostra o \autoref{code:context}.

\begin{lstlisting}[label={code:context},caption={Definição do contexto}]
type alias User =
  { name : String
  , age : Number
  }

type alias Context =
  { lastId : Number
  , users : Dict.Dict Number User
  }
\end{lstlisting}

A definição do tipo \texttt{User} poderia ser feita em outro módulo, \texttt{User.ipe}, para que todas
as suas funcionalidades (que veremos nas seções seguintes), fiquem agrupadas em um único módulo. Para
maior segurança, o tipo \texttt{User} poderia ser definido como um tipo opaco neste outro arquivo.
Deste modo, poderíamos controlar todas as maneiras de criar e acessar usuários, evitando que usuários
inválidos sejam criados, como, por exemplo, usuários com idade menor do que 0, ou cujo nome é uma
\texttt{String} vazia.

\subsection{Funções de transformação de dados}

Como a grande vantagem de usar Ipe é sua tipagem estática, e a expressividade de seus tipos, é ideal
que os tipos recebidos de fora da aplicação (como em requisições \textit{HTTP}) sejam convertidos
para tipos internos da aplicação o mais rápido possível. Similarmente, quando for necessário enviar
dados para fora da aplicação, é ideal que os tipos internos sejam convertidos para tipos externos
o mais tarde possível, apenas quando necessário. Para isso, usamos funções de transformação de dados,
que convertem os tipos internos para tipos externos, e vice-versa. No caso da aplicação base, o único
tipo externo é JSON, mas Ipe é versátil o suficiente para lidar com outros tipos de dados no futuro,
sem muita alteração no código de usuário. O \autoref{code:decoder} mostra como transformar um JSON
em um usuário.

\begin{lstlisting}[label={code:decoder},caption={Função que transforma JSON em um usuário}]
userDecoder : Json.Decode.Decoder User
userDecoder =
  Json.Decode.map2 (\name age -> { age = age, name = name })
    (Json.Decode.field 'name' Json.Decode.string )
    (Json.Decode.field 'age' Json.Decode.number)
\end{lstlisting}

Decodificadores (ou \textit{Decoders}, em inglês) são funções responsáveis por transformar tipos
externos (como JSON) em tipos internos (como \texttt{User}). Eles descrevem como traduzir os dados
para algo que Ipe entenda. No caso do \autoref{code:decoder}, o decodificador \texttt{userDecoder}
recebe um JSON e, se o JSON tiver um campo (\textit{field}, em inglês) \texttt{name} e um campo
\texttt{age}, que sejam uma \texttt{String} e um \texttt{Number}, respectivamente, ele usa esses valores
para construir um \texttt{Record} com o mesmo formato de \texttt{User}. É importante ressaltar que
esta função apenas descreve como transformar um JSON em um usuário, mas não faz a transformação em si.
Veremos que a função \texttt{Json.Decode.parseJson} é responsável por isso. A função \texttt{userDecoder}
poderia ser definida no arquivo \texttt{User.ipe}.

\begin{lstlisting}[label={code:encoder},caption={Transformando um usuário em JSON}]
encodeUser : { key: Number, value: User } -> Json.Value
encodeUser = \input ->
  Dict.empty {}
  |> Dict.insert 'id' (Json.Encode.number input.key)
  |> Dict.insert 'name' (Json.Encode.string input.value.name)
  |> Dict.insert 'age' (Json.Encode.number input.value.age)
  |> Json.Encode.object

encodeUsers : Dict.Dict Number User -> Json.Value
encodeUsers = \users ->
  users
  |> Dict.toList
  |> Json.Encode.list encodeUser
\end{lstlisting}


O \autoref{code:encoder} mostra como transformar usuários em objetos JSON, usando codificadores (ou
\texttt{encoders}, em inglês). Para facilitar a implementação da rota que retorna vários usuários,
definimos também a função \texttt{encodeUsers}, que transforma um dicionário de usuários (que é o que
temos no contexto) em um JSON. Na implementação de \texttt{encodeUser}, usamos a função
\texttt{Json.Encode.object}, que simplesmente transforma um dicionário em um objeto JSON. Estas
funções também poderiam ser definidas no arquivo \texttt{User.ipe}.

É importante notar que a representação de um usuário em JSON (usado para se comunicar com o mundo
externo) não necessariamente reflete a representação interna de um usuário, o que é desejável.
Diferentes linguagens de programação têm diferentes maneiras de representar dados, e é importante
que isto não afete a comunicação entre elas. Por exemplo, em Python, é comum usar \texttt{snake\_case},
enquanto em Ipe, é comum usar \texttt{camelCase} para nomes de variáveis. Com a estratégia das bibliotecas
de JSON de Ipe, podemos usar o nome que quisermos para as variáveis internas, e apenas mapeá-las
para o nome correto quando for necessário enviar dados para fora da aplicação. Também podemos aplicar
processamentos nos dados antes de ingeri-los para dentro de uma aplicação Ipe, como, por exemplo,
converter uma \texttt{String} para um \texttt{type union}.

\subsection{Configuração da aplicação}

Agora que já definimos o contexto e as funções de transformação de dados, podemos definir a aplicação
em si, utilizando a função \texttt{main}, que chamará a função \texttt{Http.createApp} para iniciar
um servidor que escuta chamadas \texttt{HTTP}. O \autoref{code:main-initializer} mostra a configuração
inicial da aplicação, como a porta em que o servidor deve escutar, e como construir o contexto inicial.
Exploraremos o conteúdo escondido pelas reticências na \autoref{sec:routes}.

\begin{lstlisting}[label={code:main-initializer},caption={Configuração inicial da aplicação}]
main: {} -> {}
main = \_ -> Http.createApp
    { port = 3000
    , createContext = \_ -> Promise.succeed
        { lastId = 0, users = Dict.empty {} }
    , handleRequest = \context request -> ...
    }
\end{lstlisting}

Este código diz que o servidor deve escutar na porta 3000, e que o contexto inicial deve ser um
dicionário vazio de usuários, e inicializar o \texttt{lastId} com 0. Este \texttt{Record} segue a
definição de \texttt{Context} do \autoref{code:context}.

\subsection{Rotas}\label{sec:routes}

Ipe não possui nenhum mecanismo de roteamento embutido, mas é bastante simples implementar um,
simplesmente usando \textit{pattern matching}. Por conveniência, Ipe disponibiliza o \textit{endpoint}
da URL requisitada como uma lista de \texttt{String}s, onde cada entrada da lista é um pedaço do
\textit{endpoint}. Por exemplo, a URL \texttt{/users/1} seria representada como \texttt{["users", "1"]}.
Além disso, o método da requisição é representado por uma união de tipos, como podemos ver no
\autoref{code:routes}, que substitui as reticências do \autoref{code:main-initializer}.

\begin{lstlisting}[label={code:routes},caption={Roteamento da aplicação base}]
match request.endpoint with
| ['users'] -> 
  match request.method with
    | Http.Get ->
      Promise.succeed
        { response =
          context.users
          |> encodeUsers
          |> Http.jsonResponse
        , newContext = context
        }
    
    | Http.Post ->
      match Json.Decode.parseJson userDecoder request.body with
        | Ok user ->
          Promise.succeed
            { response = 
              Dict.empty {}
              |> Dict.insert 'id' (context.lastId + 1)
              |> Json.Encode.object
              |> Http.jsonResponse 
            , newContext =
              { lastId = context.lastId + 1
              , users = Dict.insert (context.lastId + 1) user context.users 
              }
            }
        
        | Err error ->
          Promise.succeed
            { response = Http.jsonResponse (Json.Encode.string error)
            , newContext = context
            }

    | _ -> Promise.succeed { response = Http.jsonResponse (Json.Encode.string '404'), newContext = context }

| ['users', stringId] ->
  match Number.fromString stringId with
    | Nothing -> Promise.succeed { response = Http.jsonResponse (Json.Encode.string 'error: invalid ID'), newContext = context }
    | Just id ->
      match Dict.get id context.users with
        | Nothing -> Promise.succeed { response = Http.jsonResponse (Json.Encode.string 'error: user not found'), newContext = context }
        | Just user ->
          match request.method with
            | Http.Get ->
              Promise.succeed
                { response =
                  encodeUser { key = id, value = user }
                  |> Http.jsonResponse
                , newContext = context
                }
            
            | Http.Put ->
              match Json.Decode.parseJson userDecoder request.body with
                | Ok user ->
                  Promise.succeed
                    { response = 
                      Dict.empty {}
                      |> Dict.insert 'id' id
                      |> Json.Encode.object
                      |> Http.jsonResponse 
                    , newContext =
                      { lastId = context.lastId
                      , users = Dict.insert id user context.users 
                      }
                    }
        
                | Err error ->
                  Promise.succeed
                    { response = Http.jsonResponse (Json.Encode.string error)
                    , newContext = context
                    }

            
            | Http.Delete ->
              Promise.succeed
                { response = Http.jsonResponse (Json.Encode.string id)
                , newContext =
                  { lastId = context.lastId
                  , users = Dict.remove id context.users
                  }
                }
          
            | _ -> Promise.succeed { response = Http.jsonResponse (Json.Encode.string id), newContext = context }

| _ -> Promise.succeed { response = Http.jsonResponse (Json.Encode.string '404'), newContext = context }
\end{lstlisting}

A estratégia de roteamento é fazer \texttt{pattern matching} com o \textit{endpoint} da requisição,
e então fazer \texttt{pattern matching} com o método da requisição. Embora seja uma solução mais
verbosa do que as outras linguagens analisadas, tudo o que está acontecendo é bastante explícito,
a fim de ser mais fácil de compreender. Além disso, o programador é forçado a validar todos os dados
vindos do mundo exterior, o que é uma boa prática de programação. No exemplo acima, o programador
é obrigado a validar que o ID do usuário é um número, e que o usuário existe no contexto.

Com esta estratégia de roteamento, também é bastante simples adicionar a funcionalidade de
\textit{middleware}, que é uma função que é executada antes de um grupo qualquer de rotas (por exemplo,
todas as rotas de usuário). Como tudo é baseado em funções e \textit{pattern matching}, basta chamar
a função desejada no \textit{pattern matching} desejado. Esta estratégia também possibilita que o
programador divida a lógica de roteamento de qualquer maneira que quiser.

\subsection{Comparação com outras linguagens}

Dentre todas as implementações, Ipe foi a linguagem que necessitou de mais linhas escritas (um total
de 132, contando com espaços em branco), embora linguagens que usam \textit{frameworks} com geração
de código (C\#, Elixir e Haskell) para iniciar novos projetos tenham vários arquivos gerados, resultando
em um número de linhas de código total muito maior, o que pode ser intimidante para programadores
que não conhecem a linguagem ou o \textit{framework}.

Ipe também não possui nenhuma função ou decorador específico para definir rotas da aplicação, o que
pode resultar em código mais verboso, como podemos ver no \autoref{code:routes}. Porém, esta
estratégia de roteamento é bastante simples de entender, e é bastante flexível, permitindo que o
programador divida a lógica de roteamento da maneira que quiser. Além disso, em rotas com parâmetros
(como \texttt{/users/:id}), todas as outras linguagens dependem em definir uma \textit{string} com esse
parâmetro, e depois recebê-lo através de uma variável. Esta abordagem é bastante suscetível a erros
de nomeação, ou aceitar menos ou mais parâmetros do que o esperado. Em Ipe, um simples \textit{pattern
    matching} é responsável por perceber que a variável existe na URL, e capturá-la em uma variável.

É muito mais simples iniciar um projeto em Ipe do que em qualquer uma das linguagens analisadas. Todas
as outras linguagens requerem a instalação do compilador ou interpretador da linguagem, além do processo
de escolha de \textit{framework}, e a instalação do \textit{framework} escolhido, ou de sua ferramenta
de geração de código. Para iniciar um projeto de API em Ipe, basta instalar seu compilador e um ambiente
de execução Javascript, e escrever um arquivo \texttt{.ipe}, sem a necessidade de qualquer configuração
adicional.

Além disso, por conta de seu sistema de tipos, é muito difícil que um programa encontre um erro que
o faça parar de funcionar em tempo de execução, pois Ipe obriga o programador em pensar em casos de
erro. Isto é bastante diferente de linguagens sem tipagem estática (Python, Javascript e Elixir), e
também é melhor do que Haskell, que, embora sua ótima tipagem estática, ainda permite que o programador
ignore casos de entrada inválida. E, diferente de todas as outras linguagens, Ipe não possui o conceito
de erros, fazendo com que estes casos sejam lidados explicitamente, e impedindo a chance de um erro
não ser pego (geralmente com blocos \texttt{try/catch}).

Desta forma, embora Ipe possa ser mais verbosa e devagar de se escrever do que outras linguagens,
ela é mais segura, e mais simples de se iniciar um projeto. Além disso, Ipe é capaz de prover mais
garantias de funcionamento, o que é desejável para aplicações \textit{backend}, que comumente têm a
necessidade de funcionar indefinidamente. Ainda, assim como Elm, aplicações Ipe têm a tendência de
seguirem uma estrutura geral parecida com outras aplicações Ipe, o que facilita a compreensão de
código escrito por outros programadores.
