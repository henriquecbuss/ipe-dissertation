%https://github.com/SublimeText/LaTeXTools/issues/1439
%!TEX output_directory=latexcache

% You can build this using the command:
% latexmk -pdf -jobname=output -output-directory=cache -aux-directory=cache -pdflatex="pdflatex -interaction=nonstopmode" -use-make main.tex

% When the bibliography includes a cyclic reference to another bibliography,
% you need to run `pdflatex` 5 times on the following order:
% 1. `pdflatex`,
% 2. `biber`,
% 3. `pdflatex`
% 4. `pdflatex`
% 5. `pdflatex`
% 6. `biber`
% 7. `pdflatex`

% Monograph LaTeX Template for UFSC based on:
% 1. https://github.com/royertiago/tcc
% 2. http://portal.bu.ufsc.br/normalizacao/
% 3. https://github.com/evandrocoan/ufscthesisx
% 4. http://www.latextemplates.com/template/simple-sectioned-essay

% Initially translated from Portuguese with help of https://github.com/omegat-org/omegat <Computer Assisted Translation of LaTeX document>
% https://tex.stackexchange.com/questions/313732/computer-assisted-translation-of-latex-document

% Allows you to write your thesis both in English and Portuguese
% https://tex.stackexchange.com/questions/5076/is-it-possible-to-keep-my-translation-together-with-original-text
\newif\ifenglish\englishfalse
\newif\ifadvisor\advisorfalse

% Uncomment the line `\englishtrue` to set the document default language to English
% \englishtrue
\advisortrue

% https://tex.stackexchange.com/questions/131002/how-to-expand-ifthenelse-so-that-it-can-be-used-in-parshape
\newcommand{\lang}[2]{\ifenglish#1\else#2\fi}
\newcommand{\advisor}[2]{\ifadvisor#1\else#2\fi}

% https://tex.stackexchange.com/questions/385895/how-to-make-passoptionstopackage-add-the-option-as-the-last
% https://tex.stackexchange.com/questions/484400/changing-the-cleveref-package-language-conjunction-definition
% https://tex.stackexchange.com/questions/516058/why-isnt-my-biblatex-language-changing-when-passing-the-language-on-my-document
\ifenglish
    \PassOptionsToPackage{brazil,main=english,spanish,french}{babel}
\else
    \PassOptionsToPackage{main=brazil,english,spanish,french}{babel}
\fi

% Simple alias for English and Portuguese words
% https://tex.stackexchange.com/questions/513019/argument-of-bbltempd-has-an-extra
\newcommand{\brazilword}[1]{\protect\foreignlanguage{brazil}{#1}}
\newcommand{\englishword}[1]{\protect\foreignlanguage{english}{#1}}

% Allow you to write `Evandro's house` in latex as `Evandro\s house` instead of `Evandro\textquotesingle{}s house`
% https://tex.stackexchange.com/questions/31091/space-after-latex-commands
\newcommand{\s}[0]{\textquotesingle{}s\xspace}
\newcommand{\q}[0]{\textquotesingle{}\xspace}

% Uncomment the following line if you want to use other biblatex settings
% \PassOptionsToPackage{style=numeric,repeatfields=true,backend=biber,backref=true,citecounter=true}{biblatex}
\documentclass[
\lang{english}{brazilian,brazil}, % https://tex.stackexchange.com/questions/484400/changing-the-cleveref-package-language-conjunction-definition
12pt, % Padrão UFSC para versão final
a4paper, % Padrão UFSC para versão final
oneside, % Impressão nos dois lados da folha
chapter=TITLE, % Título de capítulos em caixa alta
section=TITLE, % Título de seções em caixa alta
]{setup/ufscthesisx}

% Utilize o arquivo aftertext/references.bib para incluir sua bibliografia.
% http://tug.ctan.org/tex-archive/macros/latex/contrib/cleveref/cleveref.pdf
\addbibresource{aftertext/references.bib}

% https://www.overleaf.com/learn/latex/Inserting_Images
\graphicspath{{pictures/}}

\autor{\brazilword{Henrique da Cunha Buss}}
\titulo{\lang{Ipe Programming Language}{Linguagem de Programação Ipe}}

\orientador[\lang{Supervisor}{Orientador(a)}]{\brazilword{Maicon Rafael Zatelli}, \lang{Phd.}{Dr.}}

\coordenador[\lang{Coordinator}{Coordenador(a)}]{\brazilword{Renato Cislaghi}, \lang{Phd.}{Dr.}}

\local{\brazilword{Florianópolis, Santa Catarina} -- \lang{Brazil}{Brasil}}

\ano{2022}
\biblioteca{\lang{University Library}{Biblioteca Universitária}}

\instituicaosigla{UFSC}
\instituicao{\brazilword{Universidade Federal de Santa Catarina}}

\tipotrabalho{\lang{Bachelor\s Thesis}{Trabalho de Conclusão de Curso}}

\formacao{\lang
    {Bachelor of Science degree in Computer Science}
    {Bacharel em Ciências da Computação}%
}
\programa{\lang
    {Undergraduate Program in Computer Science}
    {Programa de Graduação em Ciências da Computação}%
}

\centro{\lang
    {INE -- Department of Informatics and Statistics, CTC -- Technological Center}
    {INE -- Departamento de Informática e Estatística, CTC -- Centro Tecnológico}
}

\campus{\brazilword{Campus Reitor João David Ferreira Lima}}

% TODO: Adicionar data de defesa
% \data{\lang{30 of march of}{30 de março de} 2019}
\data{(TODO - Adicionar data de defesa)}

\preambulo{\lang%
    {%
        \imprimirtipotrabalho~submitted to the \imprimirprograma~of
        \imprimirinstituicao~for degree acquirement in \imprimirformacao.%
    }{%
        \imprimirtipotrabalho~submetido ao \imprimirprograma~da
        \imprimirinstituicao~para a obtenção do Grau de \imprimirformacao.%
    }%
}

% Allows you to use ~= instead of `\hyp{}`
% https://tex.stackexchange.com/questions/488008/how-to-create-an-alternative-to-shortcut-or-hyp
% https://tex.stackexchange.com/questions/405718/depending-on-babel-language-setting-i-get-biblatex-error-argument-of-language
% https://tex.stackexchange.com/questions/340661/argument-of-languageactivearg-has-an-extra-i-use-includegraphics-and-r
\useshorthands{~}\defineshorthand{~=}{\hyp{}}

\palavraschaveufsc{palavraschaveingles}   {Programming language}
\palavraschaveufsc{palavraschaveportugues}{Linguagem de programação}

\palavraschaveufsc{palavraschaveingles}   {Functional programming}
\palavraschaveufsc{palavraschaveportugues}{Programação funcional}

\hypersetup
{
    pdfsubject={Thesis' Abstract},
    pdfcreator={LaTeX with abnTeX2 for UFSC},
    pdftitle={\imprimirtitulo},
    pdfauthor={\imprimirautor},
    pdfkeywords={\lang{\palavraschaveinglessemitem}{\palavraschaveportuguessemitem}},
}

% Altere o arquivo 'settings.tex' para incluir customizações de aparência da sua tese
%----------------------------------------------------------------------------------------
%   Thesis Tweaks and Utilities
%----------------------------------------------------------------------------------------
\makeatletter


% Uncomment this if you are debugging pages' badness Underfull & Overflow
% https://tex.stackexchange.com/questions/115908/geometry-showframe-landscape
% https://tex.stackexchange.com/questions/387077/what-is-the-difference-between-usepackageshowframe-and-usepackageshowframe
% https://tex.stackexchange.com/questions/387257/how-to-do-the-memoir-headings-fix-but-not-have-my-text-going-over-the-page-botto
% https://tex.stackexchange.com/questions/14508/print-page-margins-of-a-document
% \usepackage[showframe,pass]{geometry}

% To use the font Times New Roman, instead of the default LaTeX font
% more up-to-date than '\usepackage{mathptmx}'
% \usepackage{newtxtext}
% \usepackage{newtxmath}

\usepackage{simplebnf}

% https://tex.stackexchange.com/questions/182569/how-to-manually-set-where-a-word-is-split
\hyphenation{Ge-la-im}
\hyphenation{Cis-la-ghi}

% Add missing translations for Portuguese
% https://tex.stackexchange.com/questions/8564/what-is-the-right-way-to-redefine-macros-defined-by-babel
\@ifpackageloaded{babel}{\@ifpackagewith{babel}{brazil}{\addto\captionsbrazil{%
      \renewcommand{\mytextpreliminarylistname}{Breve Sumário}
    }}{}}{}
\@ifundefined{advisor}{\newcommand{\advisor}[2]{#1}}{}

% Selects a sans serif font family
% \renewcommand{\sfdefault}{cmss}

% Selects a monospaced (“typewriter”) font family
% \renewcommand{\ttdefault}{cmtt}

% Spacing between lines and paragraphs
% https://tex.stackexchange.com/questions/70212/ifpackageloaded-question
\@ifclassloaded{memoir}
{
  % New custom chapter style VZ14, see other chapters styles in:
  % http://repositorios.cpai.unb.br/ctan/info/latex-samples/MemoirChapStyles/MemoirChapStyles.pdf
  \newcommand\thickhrulefill{\leavevmode \leaders \hrule height 1ex \hfill \kern \z@}
  \makechapterstyle{VZ14} { %
    % \thispagestyle{empty}
    \setlength\beforechapskip{50pt}
    \setlength\midchapskip{20pt}
    \setlength\afterchapskip{20pt}
    \renewcommand\chapternamenum{}
    \renewcommand\printchaptername{}
    \renewcommand\chapnamefont{\Huge\scshape}
    \renewcommand\printchapternum {%
      \chapnamefont\null\thickhrulefill\quad
      \@chapapp\space\thechapter\quad\thickhrulefill
    }
    \renewcommand\printchapternonum {%
      \par\thickhrulefill\par\vskip\midchapskip
      \hrule\vskip\midchapskip
    }
    \renewcommand\chaptitlefont{\huge\scshape\centering}
    \renewcommand\afterchapternum {%
      \par\nobreak\vskip\midchapskip\hrule\vskip\midchapskip
    }
    \renewcommand\afterchaptertitle {%
      \par\vskip\midchapskip\hrule\nobreak\vskip\afterchapskip
    }
  }

  % Apply the style `VZ14` just created
  % \chapterstyle{VZ14}

  % http://mirrors.ibiblio.org/CTAN/macros/latex/contrib/memoir/memman.pdf
  \setlength\beforechapskip{0pt}
  \setlength\midchapskip{15pt}
  \setlength\afterchapskip{15pt}

  % Memoir: Warnings “The material used in the headers is too large” w/ accented titles
  % https://tex.stackexchange.com/questions/387293/how-to-change-the-page-layout-with-memoir
  \setheadfoot{30.0pt}{\footskip}
  \checkandfixthelayout
}{}

% Controlling the spacing between one paragraph and another
% Default value for UFSC 0.0cm
\setlength{\parskip}{\advisor{0.0cm}{0.2cm}}

% Paragraph size is given by
% Default value for UFSC 1.5cm
% \setlength{\parindent}{1.3cm}

% https://tex.stackexchange.com/questions/148647/how-to-remove-space-before-enumerate
% https://tex.stackexchange.com/questions/433543/behaviour-of-enumitem-setlist
\advisor{}{
  \setlist*[enumerate]{label=\arabic*,}
  \setlist*[enumerateoptional]{label=\arabic*,}

  % https://tex.stackexchange.com/questions/24454/space-after-float-with-h
  % https://tex.stackexchange.com/questions/23313/how-can-i-reduce-padding-after-figure
  \AtBeginEnvironment{figure}{
    \setlength{\intextsep}{5pt} % Vertical space above & below [h] floats
    % \setlength{\textfloatsep}{10pt} % Vertical space below (above) [t] ([b]) floats
    % \setlength{\abovecaptionskip}{10pt}
    % \setlength{\belowcaptionskip}{5pt}
  }

  % Patch the `abntex2` citacao environment removing the extra space from its top
  % https://tex.stackexchange.com/questions/300340/topsep-itemsep-partopsep-and-parsep-what-does-each-of-them-mean-and-wha
  \xpatchcmd{\citacao}
  {\list{}}
  {\list{}{\topsep=0pt}}
  {}
  {\FAILEDPATCHINGCITACAO}
}


% Color settings across the document
\@ifpackageloaded{xcolor}
{
  % RGB colors in absolute values from 0 to 255 by using `RGB` tag
  \definecolor{darkblue}{RGB}{26,13,178}

  % Colors names definitions as RGB colors in percentage notation by using `rgb` tag
  \definecolor{mygreen}{rgb}{0,0.6,0}
  \definecolor{mygray}{rgb}{0.5,0.5,0.5}
  \definecolor{mymauve}{rgb}{0.58,0,0.82}
  \definecolor{figcolor}{rgb}{1,0.4,0}
  \definecolor{tabcolor}{rgb}{1,0.4,0}
  \definecolor{eqncolor}{rgb}{1,0.4,0}
  \definecolor{linkcolor}{rgb}{1,0.4,0}
  \definecolor{citecolor}{rgb}{1,0.4,0}
  \definecolor{seccolor}{rgb}{0,0,1}
  \definecolor{abscolor}{rgb}{0,0,1}
  \definecolor{titlecolor}{rgb}{0,0,1}
  \definecolor{biocolor}{rgb}{0,0,1}
  \definecolor{blue}{RGB}{41,5,195}

  % PDF Hyperlinks settings
  \@ifpackageloaded{hyperref}
  {
    \hypersetup
    {
      colorlinks=true,     % false: boxed links; true: colored links
      linkcolor=darkblue,  % color of internal links
      citecolor=darkblue, % color of links to bibliography
      filecolor=black,     % color of file links
      urlcolor=\advisor{black}{darkgreen},
      bookmarksdepth=4,
      pdfencoding=auto,%
      psdextra,
    }
  }
}{}


% Filtering and Mapping Bibliographies
% \DeclareFieldFormat{url}{Disponível~em:\addspace\url{#1}}

% https://tex.stackexchange.com/questions/517526/how-to-make-biblatex-url-links-generated-with-brackets-around-it-url-correctly
\DeclareFieldFormat{url}{\bibstring{urlfrom}\addcolon\space\textless\url{#1}\textgreater}
\DefineBibliographyStrings{brazil}{urlfrom = {Disponível em}}
\DefineBibliographyStrings{english}{urlfrom = {Available from}}

% https://tex.stackexchange.com/questions/391695/is-possible-to-remove-the-link-color-of-the-comma-on-the-citation-link
% \DeclareFieldFormat{citehyperref}{#1}

% % https://tex.stackexchange.com/questions/203764/reduce-font-size-of-bibliography-overfull-bibliography
% \newcommand{\bibliographyfontsize}{\fontsize{10.0pt}{10.5pt}\selectfont}
% \renewcommand*{\bibfont}{\bibliographyfontsize}

% Uncomment this to insert the abstract into your bibliography entries when the abstract is available
% https://tex.stackexchange.com/questions/398666/how-to-correctly-insert-and-justify-abstract
\ifadvisor
\else
  \DeclareFieldFormat{abstract}%
  {%
    \par\justifying
    \begin{adjustwidth}{1cm}{}
      \textbf{\bibsentence\bibstring{abstract}:} #1
    \end{adjustwidth}
  }
  \renewbibmacro*{finentry}%
  {%
    \iffieldundef{abstract}
    {\finentry}
    {\finentrypunct
      \printfield{abstract}%
      \renewcommand*{\finentrypunct}{}%
      \finentry
    }
  }

  % Backref package settings, pages with citations in bibliography
  \newcommand{\biblatexcitedntimes}{\autocap{c}ited \arabic{citecounter} times}
  \newcommand{\biblatexcitedonetime}{\autocap{c}ited one time}
  \newcommand{\biblatexcitednotimes}{\autocap{n}o citation in the text}

  \@ifpackageloaded{babel}{\@ifpackagewith{babel}{brazil}{\addto\captionsbrazil{%
        \renewcommand{\biblatexcitedntimes}{\autocap{c}itado \arabic{citecounter} vezes}
        \renewcommand{\biblatexcitedonetime}{\autocap{c}itado uma vez}
        \renewcommand{\biblatexcitednotimes}{\autocap{n}enhuma citação no texto}
      }}{}}{}
  \@ifpackageloaded{biblatex}
  {%
    % https://tex.stackexchange.com/questions/483707/how-to-detect-whether-the-option-citecounter-was-enabled-on-biblatex
    \ifx\blx@citecounter\relax
      \message{Is citecounter defined? NO!^^J}
    \else
      \message{Is citecounter defined? YES!^^J}
      \ifbacktracker
        \message{Is backtracker defined? YES!^^J}
        \renewbibmacro*{pageref}
        {%
          % https://tex.stackexchange.com/questions/516054/how-to-use-a-dot-to-separate-my-new-bibliography-entry
          \renewcommand*{\bibpagerefpunct}{\addperiod\space}%
          \iflistundef{pageref}%
          {\printtext{\biblatexcitednotimes}}
          {%
            \printtext
            {%
              \ifnumgreater{\value{citecounter}}{1}
              {\biblatexcitedntimes}
              {\biblatexcitedonetime}%
            }%
            \setunit{\addspace}%
            \ifnumgreater{\value{pageref}}{1}
            {\bibstring{backrefpages}\ppspace}
            {\bibstring{backrefpage}\ppspace}%
            \printlist[pageref][-\value{listtotal}]{pageref}%
          }%
        }

        \DefineBibliographyStrings{brazil}
        {
          backrefpage  = {na página},
          backrefpages = {nas páginas},
        }

        \DefineBibliographyStrings{english}
        {
          backrefpage  = {on page},
          backrefpages = {on pages},
        }
      \else
        \message{Is backtracker defined? NO!^^J}
      \fi
    \fi
  }{}
\fi


% https://tex.stackexchange.com/questions/516056/why-an-empty-or-not-biblatex-declaresourcemap-is-removing-my-bibliography-acces
% https://github.com/abntex/biblatex-abnt/pull/56/files
\DeclareStyleSourcemap{%% >>>2
  % This maps some fields used in abntex2cite to biblatex fields.
  \maps[datatype=bibtex]{%
    \map{%
      \step[fieldsource=conference-number,fieldtarget=number]%
      \step[fieldsource=conference-year,fieldtarget=eventdate]%
      \step[fieldsource=conference-location,fieldtarget=venue]%
      \step[fieldsource=conference-number,fieldtarget=number]%
      \step[fieldsource=org-short,fieldtarget=shortauthor]%
      \step[fieldsource=urlaccessdate,fieldtarget=urldate]%
      \step[fieldsource=year-presented,fieldtarget=eventyear]%
      \step[fieldsource=furtherresp,fieldtarget=titleaddon]%
      \step[typesource=journalpart,typetarget=supperiodical]%
    }%
    \map[overwrite=false]{%
      \step[fieldsource=reprinted-from, final]%
      \step[fieldset=related, origfieldval]%
    }%
    \map[overwrite=false]{%
      \step[fieldsource=reprinted-text, final]%
      \step[fieldset=relatedtype, fieldvalue={reprintfrom}]%
    }%
    \map{%
      \pertype{patent}% Use the organization as sourcekey for patents
      \step[fieldsource=organization, final]%
      \step[fieldset=sortkey, origfieldval]%
    }%
    \map[overwrite=false]{%
      \pertype{thesis}%
      \pertype{phdthesis}%
      \pertype{mastersthesis}%
      \pertype{monography}%
      \step[fieldset=bookpagination, fieldvalue={sheet}]%
    }%
    % remove fields that are always useless
    \map{
      % \step[fieldset=abstract, null]
      \step[fieldset=pagetotal, null]
    }
    % % remove URLs for types that are primarily printed
    % \map{
    %   \pernottype{software}
    %   \pernottype{online}
    %   \pernottype{report}
    %   \pernottype{techreport}
    %   \pernottype{standard}
    %   \pernottype{manual}
    %   \pernottype{misc}
    %   \step[fieldset=url, null]
    %   \step[fieldset=urldate, null]
    % }
    \map{
      \pertype{inproceedings}
      % remove mostly redundant conference information
      \step[fieldset=venue, null]
      \step[fieldset=eventdate, null]
      \step[fieldset=eventtitle, null]
      % do not show ISBN for proceedings
      \step[fieldset=isbn, null]
      % Citavi bug
      \step[fieldset=volume, null]
    }
  }%
}% <<<2


% https://tex.stackexchange.com/questions/14314/changing-the-font-of-the-numbers-in-the-toc-in-the-memoir-class
\renewcommand{\cftpartfont}{\ABNTEXpartfont\color{black}}
\renewcommand{\cftpartpagefont}{\ABNTEXpartfont\color{black}}

\renewcommand{\cftchapterfont}{\ABNTEXchapterfont\color{black}}
\renewcommand{\cftchapterpagefont}{\ABNTEXchapterfont\color{black}}

\renewcommand{\cftsectionfont}{\ABNTEXsectionfont\color{black}}
\renewcommand{\cftsectionpagefont}{\ABNTEXsectionfont\color{black}}

\renewcommand{\cftsubsectionfont}{\ABNTEXsubsectionfont\color{black}}
\renewcommand{\cftsubsectionpagefont}{\ABNTEXsubsectionfont\color{black}}

\renewcommand{\cftsubsubsectionfont}{\ABNTEXsubsubsectionfont\color{black}}
\renewcommand{\cftsubsubsectionpagefont}{\ABNTEXsubsubsectionfont\color{black}}

\renewcommand{\cftparagraphfont}{\ABNTEXsubsubsubsectionfont\color{black}}
\renewcommand{\cftparagraphpagefont}{\ABNTEXsubsubsubsectionfont\color{black}}

% Memoir has another mechanism for the job: \cftsetindents{‹kind›}{indent}{numwidth}. Here kind is
% chapter, section, or whatever; the indent specifies the ‘margin’ before the entry starts; and the
% width is of the box into which the number is typeset (so needs to be wide enough for the largest
% number, with the necessary spacing to separate it from what comes after it in the line.
% http://www.tex.ac.uk/FAQ-tocloftwrong.html
% https://tex.stackexchange.com/questions/264668/memoir-indentation-of-unnumbered-sections-in-table-of-contents
% https://tex.stackexchange.com/questions/394227/memoir-toc-indent-the-second-line-by-numberspace
%
% `\cftlastnumwidth` and these `\cftsetindents` are defined by the abntex2 class,
% obeying the `ABNTEXsumario-abnt-6027-2012`. \newlength{\cftlastnumwidth}
% \setlength{\cftlastnumwidth}{\cftsubsubsectionnumwidth}
% \addtolength{\cftlastnumwidth}{-1em}

% http://www.tex.ac.uk/FAQ-tocloftwrong.html
% Use \setlength\cftsectionnumwidth{4em} to override all these values at once
\ifadvisor
\else
  \makechapterstyle{fixedabntex2indentation}
  {%
    \renewcommand{\chapterheadstart}{}
    \setlength{\beforechapskip}{20pt}
    \setlength{\midchapskip}{20pt}
    \setlength{\afterchapskip}{15pt}

    \ifx \chapternamenumlength \undefined
      \newlength{\chapternamenumlength}
    \fi

    % tamanhos de fontes de chapter e part
    \ifthenelse{\equal{\ABNTEXisarticle}{true}}{%
      \setlength\beforechapskip{\baselineskip}%
      \renewcommand{\chaptitlefont}{\ABNTEXsectionfont\ABNTEXsectionfontsize}%
    }{%else
      \setlength{\beforechapskip}{0pt}%
      \renewcommand{\chaptitlefont}{\ABNTEXchapterfont\ABNTEXchapterfontsize}%
    }

    \renewcommand{\chapnumfont}{\chaptitlefont}
    \renewcommand{\parttitlefont}{\ABNTEXpartfont\ABNTEXpartfontsize}
    \renewcommand{\partnumfont}{\ABNTEXpartfont\ABNTEXpartfontsize}
    \renewcommand{\partnamefont}{\ABNTEXpartfont\ABNTEXpartfontsize}

    % tamanhos de fontes de section, subsection, subsubsection e subsubsubsection
    \setsecheadstyle{\ABNTEXsectionfont\ABNTEXsectionfontsize\ABNTEXsectionupperifneeded}
    \setsubsecheadstyle{\ABNTEXsubsectionfont\ABNTEXsubsectionfontsize\ABNTEXsubsectionupperifneeded}
    \setsubsubsecheadstyle{\ABNTEXsubsubsectionfont\ABNTEXsubsubsectionfontsize\ABNTEXsubsubsectionupperifneeded}
    \setsubsubsubsecheadstyle{\ABNTEXsubsubsubsectionfont\ABNTEXsubsubsubsectionfontsize\ABNTEXsubsubsubsectionupperifneeded}

    % Impressão do número do capítulo
    \renewcommand{\chapternamenum}{}

    % Impressão do nome do capítulo
    \renewcommand{\printchaptername}{%
      \chaptitlefont%
      \ifthenelse{\boolean{abntex@apendiceousecao}}{\appendixname}{}%
    }

    % Impressão do título do capítulo
    \def\printchaptertitle##1{%
      \chaptitlefont%
      \ifthenelse{\boolean{abntex@innonumchapter}}{\centering\ABNTEXchapterupperifneeded{##1}}{%
        \ifthenelse{\boolean{abntex@apendiceousecao}}{%
          \centering%
          \settowidth{\chapternamenumlength}{\printchaptername\printchapternum\afterchapternum}%
          \ABNTEXchapterupperifneeded{##1}%
        }{%
          \settowidth{\chapternamenumlength}{\printchaptername\printchapternum\afterchapternum}%
          \parbox[t]{\columnwidth-\chapternamenumlength}{\ABNTEXchapterupperifneeded{##1}}}%
      }%
    }

    % https://tex.stackexchange.com/questions/264668/memoir-indentation-of-unnumbered-sections-in-table-of-contents
    \renewcommand{\tocinnonumchapter}{%
      \addtocontents{toc}{\cftsetindents{chapter}{2.5em}{2em}}%
      \cftinserthook{toc}{A}}

    % Impressão do número do capítulo (no capítulo e não toc)
    \renewcommand{\printchapternum}{%
      \setboolean{abntex@innonumchapter}{false}%
      \chapnumfont%
      ~~\thechapter~%
      \ifthenelse{\boolean{abntex@apendiceousecao}}{%
        \tocinnonumchapter%
        ~\ABNTEXcaptiondelim~~%
      }{}%
    }

    \renewcommand{\ABNTEXcaptiondelim}{~\textendash~}
    \renewcommand{\afterchapternum}{}

    % Impressão do capítulo não numerado
    \renewcommand\printchapternonum{%
      \setboolean{abntex@innonumchapter}{true}%
    }
  }
  \chapterstyle{fixedabntex2indentation}

  \cftsetindents{part}          {0em} {3em}
  \cftsetindents{chapter}       {0em} {3em}
  \cftsetindents{section}       {0em} {4.3em}
  \cftsetindents{subsection}    {0em} {5.2em}
  \cftsetindents{subsubsection} {0em} {5.1em}
  \cftsetindents{paragraph}     {0em} {6.0em}
  \cftsetindents{subparagraph}  {0em} {7.0em}
\fi


\makeatother



% When writing a large document, it is sometimes useful to work on selected sections of the document
% to speed up compilation time: https://en.wikibooks.org/wiki/TeX/includeonly
\newif\ifforcedinclude\forcedincludefalse

% \addtoincludeonly{beforetext/agradecimentos}
% \addtoincludeonly{beforetext/epigrafe}
% \addtoincludeonly{beforetext/fichacatalografica}
% \addtoincludeonly{beforetext/folhadeaprovacao}
% \addtoincludeonly{beforetext/resumos}
% \addtoincludeonly{beforetext/siglas}
% \addtoincludeonly{beforetext/simbolos}

% Part 1
% \addtoincludeonly{chapters/introduction}
% \addtoincludeonly{chapters/motivation}
% \addtoincludeonly{chapters/beautifiers}

% Part 2
\addtoincludeonly{chapters/object_beautifier}
% \addtoincludeonly{chapters/conclusion}
% \addtoincludeonly{aftertext/aftertext}

% Control whether the full document will be generated
% Note: It will also generate severals errors like the following, which can be ignored
%       Latexmk: Missing input file: 'chapters/test.aux'
%
% You can make latex stop generate these errors, if you generate a full version
% of the document, before uncommenting these lines.
%
% Uncomment these two lines, to only partially generate the document
% \doincludeonly
% \forcedincludetrue


% https://tex.stackexchange.com/questions/85113/xrightarrow-text
\makeatletter
\newcommand{\xRightarrow}[2][]{\ext@arrow 0359\Rightarrowfill@{#1}{#2}}
\newcommand{\xLeftarrow}[2][]{\ext@arrow 0359\Leftarrowfill@{#1}{#2}}
\makeatother

% https://tex.stackexchange.com/questions/32208/footnote-runs-onto-second-page
\interfootnotelinepenalty=10000

% Disable the empty pages automatically put by memoir class, except the ones by \cleardoublepage
\ifforcedinclude\openany\else\fi

% https://tex.stackexchange.com/questions/171999/overfull-hbox-in-biblatex
% https://tex.stackexchange.com/questions/499457/why-my-document-is-not-hyphenation-on-words-starting-with-upper-case-letter-i
\emergencystretch=5em

% https://tex.stackexchange.com/questions/23313/how-can-i-reduce-padding-after-figure
% https://tex.stackexchange.com/questions/499580/how-to-keep-my-default-floating-environment-spacing-before-them-while-reducing
% \xpretocmd{\figure}{\setlength{\belowcaptionskip}{-10pt}}{}{}


\begin{document}
% FIXME: Comment this after finishing the thesis, so you can start fixing the \flushbottom vs \raggedbottom
% https://tex.stackexchange.com/questions/65355/flushbottom-vs-raggedbottom
\raggedbottom

% https://tex.stackexchange.com/questions/4705/double-space-between-sentences
\frenchspacing

% Uncomment this to put a ←← | ← (Go To Top/Go Back) on each section header
\advisor{}{\addGoToSummary}

% ELEMENTOS PRÉ-TEXTUAIS


% ELEMENTOS PRÉ-TEXTUAIS
\ifforcedinclude\else
% Fix the \textpreliminarycontents not showing up when @twoside is disabled
\newif\ifufscThesisXisMemoirTwoSidesEnabled

    % https://tex.stackexchange.com/questions/360785/how-do-i-check-if-a-document-is-oneside-or-twoside
    \ifthenelse{\boolean{@twoside}}{%
        \ufscThesisXisMemoirTwoSidesEnabledtrue%
    }{%
        \ufscThesisXisMemoirTwoSidesEnabledfalse%
    }%
    \setboolean{@twoside}{true}

    % pretextual settings
    % https://tex.stackexchange.com/questions/386446/how-to-fix-destination-with-the-same-identifier-namepage-a-has-been-already
    % https://tex.stackexchange.com/questions/67989/pdftex-warning-has-been-referenced-but-does-not-exist-replaced-by-a-fixed-one
    \hypersetup{pageanchor=false}
    \PRIVATEbookmarkthis{Capa}
    \addtotextpreliminarycontent{Capa}
    \pretextual

    % Capa
    % \includepdf{pictures/FrenteCapaUFSC.pdf}
    % https://tex.stackexchange.com/questions/227711/blank-page-after-titlingpage
    \advisor{}{\AtBeginShipoutNext{\AtBeginShipoutNext{\AtBeginShipoutDiscard}}}
    \imprimircapa

    % https://tex.stackexchange.com/questions/386446/how-to-fix-destination-with-the-same-identifier-namepage-a-has-been-already
    % https://tex.stackexchange.com/questions/67989/pdftex-warning-has-been-referenced-but-does-not-exist-replaced-by-a-fixed-one
    \hypersetup{pageanchor=true}

    % Custom list throw LaTeX Error: Command \mycustomfiction already defined?
    % https://tex.stackexchange.com/questions/388489/custom-list-throw-latex-error-command-mycustomfiction-already-defined/
    \advisor{}{%
        % Manually add the `\textpreliminarycontents` to the Table of Contents here
        % to keep the hyper link pointing to the beginning of the page, instead of
        % the beginning of `\textpreliminarycontents`
        % https://tex.stackexchange.com/questions/44088/when-do-i-need-to-invoke-phantomsection
        \phantomsection\addcontentsline{toc}{chapter}{\mytextpreliminarylistname}

        % https://tex.stackexchange.com/questions/234398/list-of-figures-and-tables-when-there-are-no-figures-or-tables
        \whenlistisnotempty{\mytextpreliminarylistname}{%
            \begin{KeepFromToc}
                \textpreliminarycontents
            \end{KeepFromToc}
        }

        \clearpage
    }

    % Fix the \textpreliminarycontents not showing up when @twoside is disabled
    \ifufscThesisXisMemoirTwoSidesEnabled
        \setboolean{@twoside}{true}
    \else
        \setboolean{@twoside}{false}
    \fi

    % Folha de rosto (o * indica que haverá a ficha bibliográfica)
    % https://tex.stackexchange.com/questions/74439/table-of-contents-incorrect-page-numbering
    \addtotextpreliminarycontent{\folhaderostoname}
    \imprimirfolhaderosto{}

    \ifforcedinclude\else\cleardoublepage\fi
\fi

\newpage
\thispagestyle{empty}
\mbox{}
\newpage

% Inserir folha de aprovação. Isto é um exemplo de Folha de aprovação, elemento obrigatório da
% NBR 14724/2011 (seção 4.2.1.3). Você pode utilizar este modelo até a aprovação do trabalho.
% Após isso, substitua todo o conteúdo deste arquivo por uma imagem da página assinada pela
% banca com o comando abaixo:
\ifforcedinclude\else\cleardoublepage\fi


\addtotextpreliminarycontent{\lang{Approval Sheet}{Folha de Aprovação}}

\begin{folhadeaprovacao}

    \begin{center}
        {\imprimirautor}

        \begin{center}
            \ABNTEXchapterfont\bfseries\MakeUppercase{\imprimirtitulo}\ifnotempty{\imprimirsubtitulo}{: \imprimirsubtitulo}
        \end{center}

        \begin{minipage}{\textwidth}
            \lang
            {
                This \imprimirtipotrabalho~was considered appropriate to get the \imprimirformacao,
                \ifnotempty{\imprimirarea}{in the area of \imprimirarea,}
                and it was approved by the \imprimirprograma~of \imprimircentro~of \imprimirinstituicao.
            }
            {
                Este(a) \imprimirtipotrabalho~foi julgado adequado(a) para obtenção do Título de \imprimirformacao,
                \ifnotempty{\imprimirarea}{na área de concentração de \imprimirarea,}
                e foi aprovado em sua forma final pelo \imprimirprograma~
                do \imprimircentro~da \imprimirinstituicao.
            }
        \end{minipage}%
    \end{center}

    \begin{center}
        \imprimirlocal, \imprimirdata.
    \end{center}

    \assinatura{%
        \textbf{\imprimircoordenador} \\
        \imprimircoordenadorRotulo~\lang{of}{do} \imprimirprograma
    }

    % \newpage
    \begin{flushleft}
        \textbf{\lang{Examination Board}{Banca Examinadora}:}
    \end{flushleft}

    \assinatura{%
        \textbf{\imprimirorientador} \\ \imprimirorientadorRotulo\\
        \imprimirinstituicao~--~\imprimirinstituicaosigla
    }

    \ifnotempty{\imprimircoorientador}{%
        \assinatura{%
            \textbf{\imprimircoorientador} \\ \imprimircoorientadorRotulo \\
            \imprimirinstituicao~--~\imprimirinstituicaosigla
        }
    }

    \assinatura{%
        \textbf{Prof. Convidado 1, \lang{PhD.}{Dr.}} \\
        Instituição 1 -- Sigla 1
    }

    \assinatura{%
        \textbf{Prof. Convidado 2, \lang{PhD.}{Dr.}} \\
        Instituição 2 -- Sigla 2
    }

    \assinatura{%
        \textbf{Prof. Convidado 3, \lang{PhD.}{Dr.}} \\
        Instituição 3 -- Sigla 3
    }

    \assinatura{%
        \textbf{Prof. Convidado 4, \lang{PhD.}{Dr.}} \\
        Instituição 4 -- Sigla 4
    }

\end{folhadeaprovacao}



\newpage
\thispagestyle{empty}
\mbox{}
\newpage

% Agradecimentos
\ifforcedinclude\else\cleardoublepage\fi


% TODO - Adicionar agradecimentos
% \addtotextpreliminarycontent{\lang{Acknowledgement}{Agradecimentos}}

% \begin{agradecimentos}

%     (TODO - Adicionar agradecimentos)

% \end{agradecimentos}


\newpage
\thispagestyle{empty}
\mbox{}
\newpage

% Ajusta o espaçamento dos parágrafos do resumo
\setlength{\absparsep}{18pt}

% RESUMOS
\ifforcedinclude\else\cleardoublepage\fi


\newcommand{\imprimirbrazilabstract}{%
    \cleardoublepage\phantomsection
    \addtotextpreliminarycontent{Resumo em Português}
    \begin{otherlanguage*}{brazil}
        \begin{resumo}[Resumo]

            (TODO - Adicionar resumo)

            \imprimirpalavraschave{Palavras-chaves}{\begin{inparaitem}[]\palavraschaveportugues\end{inparaitem}}

        \end{resumo}
    \end{otherlanguage*}
}


\newcommand{\imprimirenglishabstract}{%
    \cleardoublepage\phantomsection
    \addtotextpreliminarycontent{English's Abstract}
    \begin{otherlanguage*}{english}
        \begin{resumo}[Abstract]

            (TODO - Adicionar abstract)

            \imprimirpalavraschave{Keywords}{\begin{inparaitem}[]\palavraschaveingles\end{inparaitem}}

        \end{resumo}
    \end{otherlanguage*}
}


\imprimirbrazilabstract
\imprimirenglishabstract


% Some tables of contents
\ifforcedinclude\else
    {
        % https://tex.stackexchange.com/questions/179506/disable-colorlinks-locally-or-just-for-the-toc
        \hypersetup{hidelinks}

        % inserir lista de figuras
        \ifforcedinclude\else\cleardoublepage\fi
        % https://tex.stackexchange.com/questions/234398/list-of-figures-and-tables-when-there-are-no-figures-or-tables
        \whenlistisnotempty{\listfigurename}{%
            \addtotextpreliminarycontent{\listfigurename}
            % https://tex.stackexchange.com/questions/121879/remove-spacing-between-per-chapter-figures-in-lof
            {\renewcommand{\addvspace}[1]{}
                \listoffigures*}
        }{\pdfbookmark[0]{\listfigurename}{lof}}

        % inserir lista de quadros
        \ifforcedinclude\else\cleardoublepage\fi
        % https://tex.stackexchange.com/questions/234398/list-of-figures-and-tables-when-there-are-no-figures-or-tables
        \whenlistisnotempty{\listofquadrosname}{%
            \addtotextpreliminarycontent{\listofquadrosname}
            % https://tex.stackexchange.com/questions/121879/remove-spacing-between-per-chapter-figures-in-lof
            {\renewcommand{\addvspace}[1]{}
                \listofquadros*}
        }{\pdfbookmark[0]{\listofquadrosname}{loq}}

        % inserir lista de tabelas
        \ifforcedinclude\else\cleardoublepage\fi
        % https://tex.stackexchange.com/questions/234398/list-of-figures-and-tables-when-there-are-no-figures-or-tables
        \whenlistisnotempty{\listtablename}{%
            \addtotextpreliminarycontent{\listtablename}
            % https://tex.stackexchange.com/questions/121879/remove-spacing-between-per-chapter-figures-in-lof
            {\renewcommand{\addvspace}[1]{}
                \listoftables*}
        }{\pdfbookmark[0]{\listtablename}{lot}}

        % inserir códigos fonte (List of Listings `lol`)
        % https://tex.stackexchange.com/questions/511519/latex-keeps-showing-minted-environment-as-figures-instead-of-listening/511579#511579
        \ifforcedinclude\else\cleardoublepage\fi
        % https://tex.stackexchange.com/questions/234398/list-of-figures-and-tables-when-there-are-no-figures-or-tables
        \whenlistisnotempty{\lstlistlistingname}{%
            \addtotextpreliminarycontent{\lstlistlistingname}
            % https://tex.stackexchange.com/questions/121879/remove-spacing-between-per-chapter-figures-in-lof
            {\renewcommand{\addvspace}[1]{}
                \lstlistoflistings*}
        }{\pdfbookmark[0]{\lstlistlistingname}{lol}}
    }
\fi


% inserir lista de abreviaturas e siglas
\ifforcedinclude\else\cleardoublepage\fi


\addtotextpreliminarycontent{\lang{List of Acronyms}{Lista de Siglas}}

\begin{siglas}
    \item[\textit{API}] \textit{Application Programming Interface}
    \item[\textit{REST}] \textit{Representational State Transfer}
    \item[\textit{AST}] \textit{Abstract Syntax Tree}
    \item[\textit{CRUD}] \textit{Create, Read, Update, Delete}
    \item[\textit{IOT}] \textit{Internet of Things}
    \item[\textit{OO}] \textit{Orientação a Objetos}
    \item[\textit{EBNF}] \textit{Extended Backus-Naur Form}
\end{siglas}



% Inserir lista de símbolos
\ifforcedinclude\else\cleardoublepage\fi


\addtotextpreliminarycontent{\lang{List of Symbols}{Lista de Símbolos}}

% Devam aparecer na mesma ordem de ocorrência no texto.
\begin{simbolos}
    \item[TODO] Adicionar Símbolos
\end{simbolos}


% How to remove the self-reference of the ToC from the ToC?
% https://tex.stackexchange.com/questions/10943/how-to-remove-the-self-reference-of-the-toc-from-the-toc
\ifforcedinclude\else\cleardoublepage\fi

\begin{KeepFromToc}
    % https://tex.stackexchange.com/questions/35/what-does-overfull-hbox-mean
    % https://tex.stackexchange.com/questions/59122/how-to-avoid-using-sloppy-document-wide-to-fix-overfull-hbox-problems
    % https://tex.stackexchange.com/questions/257007/adding-color-to-table-of-contents-and-section-headings
    {
        % https://tex.stackexchange.com/questions/179506/disable-colorlinks-locally-or-just-for-the-toc
        \hypersetup{hidelinks}

        % https://tex.stackexchange.com/questions/65711/underfull-vbox-badness-10000-with-memoir
        \raggedbottom

        % https://tex.stackexchange.com/questions/49887/overfull-hbox-warning-for-toc-entries-when-using-memoir-documentclass
        % \makeatletter
        % \renewcommand{\@pnumwidth}{2em}
        % \renewcommand{\@tocrmarg}{3em}
        % \makeatother

        % https://tex.stackexchange.com/questions/57544/memoir-mysterious-overfull-hbox-in-toc-when-mathptmx-is-used
        % \setlength{\cftchapternumwidth}{2.25em}

        % Add the table of contents to the brief table of contents
        % https://tex.stackexchange.com/questions/234398/list-of-figures-and-tables-when-there-are-no-figures-or-tables
        \whenlistisnotempty{\contentsname}{%
            \addtotextpreliminarycontent{\contentsname}
            \tableofcontents
        }{\pdfbookmark[0]{\contentsname}{toc}}
    }

\end{KeepFromToc}



% ELEMENTOS TEXTUAIS
\textual
\setlength\beforechapskip{50pt}
\setlength\midchapskip{20pt}
\setlength\afterchapskip{20pt}

\phantomsection

\chapter{Introdução}
\phantomsection

Hoje em dia, a maior parte do mundo depende de aplicativos, sejam eles \textit{mobile},
\textit{web} ou \textit{desktop}. A maior parte desses aplicativos depende de uma aplicação
\textit{backend} para guardar dados, conversar com outras aplicações, realizar
pagamentos. Aplicações modernas permitem que tenhamos acesso a bancos, entretenimento,
redes sociais, \textit{e-commerce}, e muito mais.

Nos últimos anos, os desenvolvedores de \textit{software} têm sido atraídos cada vez mais
ao paradigma de programação funcional. Em contraste com o paradigma de orientação
a objetos, o paradigma funcional visa simplificar o modelo mental dos programadores,
focando em funções puras, imutabilidade, e composição de funções (\textcite{functionalprogramming-future}).
Grandes linguagens de programação orientadas a objetos também estão indo na direção
da programação funcional, como Java, JavaScript, C\#, e Python. Embora esses
conceitos possam criar programas mais robustos e mais fáceis de testar, pode ser
difícil para desenvolvedores acostumados com outros paradigmas a aprenderem esse
novo modelo mental \cite{promisesoffp}.

O paradigma funcional não é algo concreto - é um conjunto de ideias que podem ser
adotadas em partes. Enquanto existem linguagens que implementam
algumas ideias da programação funcional (como as mencionadas acima), outras
tentam implementar o máximo possível - comumente chamadas de \textit{linguagens puras},
ou \textit{linguagens puramente funcionais}. Não existe uma definição exata para
o que é uma linguagem puramente funcional, mas elas geralmente têm alguns recursos,
como \textit{pattern matching}, funções puras (sem efeitos colaterais, como acesso
ao sistema de arquivos, ou mostrar uma mensagem na tela), tipagem forte e estática,
e inferência de tipos. Alguns exemplos de linguagens puramente funcionais são
\textit{Haskell} \cite{conceptionoffunctionalpl} e \textit{Elm} \cite{czaplicki2012elm}.

Linguagens de programação geralmente são criadas para um propósito específico. Por
exemplo, \textit{Javascript} foi criada para melhorar a experiência dos usuários
na \textit{web}. \textit{Elm} foi criada para simplificar a criação de aplicativos
\textit{web}, utilizando programação funcional para diminuir erros em tempo de
execução, e aumentar a produtividade dos desenvolvedores. Neste trabalho, vamos
explorar algumas das linguagens de programação utilizadas para criar aplicativos
\textit{backend} e, inspirados por \textit{Elm}, vamos criar
\textit{Ipe}\footnote{\textit{Elm} significa Olmo (uma árvore) em inglês. Ipê é uma árvore brasileira. Para simplificar o uso em inglês, vamos usar o nome \textit{Ipe}, sem acento.},
uma linguagem de programação puramente funcional, desenvolvida especificamente para
desenvolver aplicativos \textit{backend}, com o objetivo de simplificar o desenvolvimento
e diminuir a quantidade de erros em tempo de execução, além de diminuir a barreira
de entrada a linguagens puramente funcionais.

\section{Objetivos}\label{sec:objectives}

Com a devida contextualização, os objetivos deste trabalho são:

\subsection{Objetivos Gerais}

Analisar algumas linguagens de programação utilizadas para criar aplicativos
\textit{backend}, para coletar algumas práticas comuns, além de identificar
possíveis problemas ou dificuldades. Com isso, vamos ter uma base
para criar \textit{Ipe}, com o objetivo de testar o paradigma funcional para o
desenvolvimento de aplicações \textit{backend}, especificamente para APIs REST,
que é como a maioria das aplicações \textit{web} se comunicam entre si.

Além disso, Ipe visa ser uma linguagem simples de se usar, com o objetivo de
diminuir a barreira de entrada a linguagens funcionais.

\subsection{Objetivos Específicos}

Para cumprir o objetivo geral, precisamos cumprir os seguintes objetivos específicos:

\begin{enumerate}
      \item Definir os requisitos para uma aplicação \textit{backend} com uma API
            REST, para servir de modelo.
      \item Implementar esses requisitos em algumas linguagens de programação,
            procurando por padrões, boas práticas, e dificuldades.
      \item Definir a linguagem \textit{Ipe}.
      \item Criar um compilador para \textit{Ipe}, que possibilite gerar código
            que possa ser executado.
      \item Implementar os requisitos definidos no primeiro item usando \textit{Ipe},
            para comparar com as implementações descritas no item 2.
      \item Analisar e discutir os resultados obtidos.
\end{enumerate}


\section{Escopo do Trabalho}

Por questões de limitação de tempo, a aplicação modelo será bem simples, e não
necessariamente representará uma aplicação real. Além disso, \textit{Ipe} é apenas
uma prova de conceito. Não é o objetivo deste trabalho criar uma linguagem de
programação completa, com otimizações, ferramentas, e bibliotecas extensas. O
objetivo é explorar as ideias da programação funcional no desenvolvimento
\textit{backend}. Por conta disso, a linguagem \textit{Ipe} não deve ser usada
no mundo real, em projetos reais. Dito isso, ainda queremos representar um caso
de uso comum para sistemas \textit{backend}, e é por isso que temos o objetivo
de construir uma API REST em Ipe.

O compilador de \textit{Ipe} deve funcionar em sistema operacional macOS, onde
vai ser desenvolvido, e pode não funcionar em outros sistemas.

\section{Método de Pesquisa}

Para cumprir os objetivos propostos, o desenvolvimento é dividido em 3 fases, que
são explicadas a seguir:

\subsection{Trabalhos Relacionados}

Para termos noção do que outras linguagens fazem para o desenvolvimento de
aplicações \textit{backend}, vamos explorar algumas linguagens de programação
ranqueadas como mais populares em índices como o índice \textcite{tiobeindex}, e analisar algumas
linguagens que inspiraram ou influenciaram no desenvolvimento de \textit{Ipe}.

\subsection{Implementação}

Nesta fase, vamos de fato implementar \textit{Ipe} - seu \textit{parser}, compilador
e tudo o que for necessário para executar programas escritos em \textit{Ipe}.
Vamos discutir decisões de sintaxe, funções padrões, estrutura de projetos.

Mostraremos como usar \textit{Ipe}, descrevendo estruturas de controle, sistema
de tipos, estruturas de dados, funções.

\subsection{Uso em Aplicações}

Vamos desenvolver a aplicação modelo utilizando \textit{Ipe}, além de outras
pequenas aplicações de exemplo. Vamos discutir as vantagens e desvantagens de
usar \textit{Ipe} para desenvolver aplicações \textit{backend}, e comparar os
resultados obtidos com \textit{Ipe} e os resultados obtidos com linguagens orientadas
a objetos, e também com outras linguagens funcionais.

\subsection{Avaliação dos resultados}

Após implementar a linguagem e a aplicação base, também conduziremos sessões
de entrevista onde coletaremos \textit{feedback} de outros desenvolvedores para
obter métricas de avaliação. Nessas sessões de entrevista, desenvolvedores serão
convidados a desenvolver a mesma aplicação base em Ipe. Ao final das entrevistas,
os convidados irão responder perguntas sobre vantagens e desvantagens de Ipe em
relação a outras linguagens, e poderemos ter métricas de comparação.


\phantomsection

\chapter{Método de Pesquisa}
\phantomsection



% The \phantomsection command is needed to create a link to a place in the document that is not a
% figure, equation, table, section, subsection, chapter, etc.
% https://tex.stackexchange.com/questions/44088/when-do-i-need-to-invoke-phantomsection
\phantomsection

% Multiple-language document - babel - selectlanguage vs begin/end{otherlanguage}
% https://tex.stackexchange.com/questions/36526/multiple-language-document-babel-selectlanguage-vs-begin-endotherlanguage
\begin{otherlanguage*}{brazil}

    \chapter{Titulo do Capitulo}

    \begin{flushright}
        \englishword{\showfont}
    \end{flushright}


    % \newpage
    \section{Título da Seção}

    Lipsum me [1]

    \englishword{\showfont}

    Lipsum me [2-3]

\end{otherlanguage*}



% Conclusão (outro exemplo de capítulo sem numeração e presente no sumário)

% The \phantomsection command is needed to create a link to a place in the document that is not a
% figure, equation, table, section, subsection, chapter, etc.
% https://tex.stackexchange.com/questions/44088/when-do-i-need-to-invoke-phantomsection
\phantomsection

% ---
\chapter{\lang{Final Remarks}{Considerações Finais}}
\phantomsection

Lipsum me [31-33]



% ELEMENTOS PÓS-TEXTUAIS
\postextual
\setlength\beforechapskip{0pt}
\setlength\midchapskip{15pt}
\setlength\afterchapskip{15pt}

% Referências bibliográficas
\begingroup
% https://tex.stackexchange.com/questions/163559/how-to-modify-line-spacing-per-entry-of-bibliography
% https://tex.stackexchange.com/questions/19105/how-can-i-put-more-space-between-bibliography-entries-biblatex
\setlength\bibitemsep{\baselineskip}
\advisor{}{\linespread{1.18}\selectfont}

% https://tex.stackexchange.com/questions/17128/using-bibtex-to-make-a-list-of-references-without-having-citations-in-the-body
% \nocite{*}
\printbibliography[title=\lang{REFERENCES}{REFERÊNCIAS}]
\endgroup

% TODO - Adicionar apêndices
% % Inicia os apêndices
% \begin{apendicesenv}
%     % Imprime uma página indicando o início dos apêndices
%     \ifforcedinclude\else\partapendices\fi
%     \setlength\beforechapskip{50pt}
%     \setlength\midchapskip{20pt}
%     \setlength\afterchapskip{20pt}

%     


%
% How to fix the Underfull \vbox badness has occurred while \output is active on my memoir chapter style?
% https://tex.stackexchange.com/questions/387881/how-to-fix-the-underfull-vbox-badness-has-occurred-while-output-is-active-on-m
%

% ---

\lang
{\chapter[Page not filled]{Since this page is not being completely filled, it is generating the bottom bottom of the page}}
{\chapter[Página não gerada]{Como esta página não está sendo completamente preenchida, ele está gerando a caixa inferior inferior da página}}
% ---


% Multiple-language document - babel - selectlanguage vs begin/end{otherlanguage}
% https://tex.stackexchange.com/questions/36526/multiple-language-document-babel-selectlanguage-vs-begin-endotherlanguage
\begin{otherlanguage*}{english}

    \showfont

    1. How to display the font size in use in the final output,
    2. How to display the font size in use in the final output,
    3. How to display the font size in use in the final output,
    4. How to display the font size in use in the final output,
    5. How to display the font size in use in the final output,
    6. How to display the font size in use in the final output,
    7. How to display the font size in use in the final output,
    8. How to display the font size in use in the final output,
    9. How to display the font size in use in the final output,


    % As this page is not being completely filled, it is generating the page bottom bad box.
    % Fix Underfull \vbox (badness 10000) has occurred while \output is active
    %
    % \flushbottom vs \raggedbottom
    % https://tex.stackexchange.com/questions/65355/flushbottom-vs-raggedbottom
    \newpage



    \section[Some encoding tests]{\showfont}

    1. How to display the font size in use in the final output,
    2. How to display the font size in use in the final output,
    3. How to display the font size in use in the final output,
    4. How to display the font size in use in the final output,
    5. How to display the font size in use in the final output,
    6. How to display the font size in use in the final output,

    7. How to display the font size in use in the final output,
    8. How to display the font size in use in the final output,
    9. How to display the font size in use in the final output,
    10. How to display the font size in use in the final output,
    11. How to display the font size in use in the final output,
    12. How to display the font size in use in the final output,

    \subsection{\showfont}

    1. How to display the font size in use in the final output,
    2. How to display the font size in use in the final output,
    3. How to display the font size in use in the final output,
    4. How to display the font size in use in the final output,
    5. How to display the font size in use in the final output,
    6. How to display the font size in use in the final output,

    7. How to display the font size in use in the final output,
    8. How to display the font size in use in the final output,
    9. How to display the font size in use in the final output,
    10. How to display the font size in use in the final output,
    11. How to display the font size in use in the final output,
    12. How to display the font size in use in the final output,

    \subsubsection{\showfont}

    1. How to display the font size in use in the final output,
    2. How to display the font size in use in the final output,
    3. How to display the font size in use in the final output,
    4. How to display the font size in use in the final output,
    5. How to display the font size in use in the final output,
    6. How to display the font size in use in the final output,

    7. How to display the font size in use in the final output,
    8. How to display the font size in use in the final output,
    9. How to display the font size in use in the final output,
    10. How to display the font size in use in the final output,
    11. How to display the font size in use in the final output,
    12. How to display the font size in use in the final output,

    \subsubsubsection{\showfont}

    1. How to display the font size in use in the final output,
    2. How to display the font size in use in the final output,
    3. How to display the font size in use in the final output,
    4. How to display the font size in use in the final output,
    5. How to display the font size in use in the final output,
    6. How to display the font size in use in the final output,
    7. How to display the font size in use in the final output,

    8. How to display the font size in use in the final output,
    9. How to display the font size in use in the final output,
    10. How to display the font size in use in the final output,
    11. How to display the font size in use in the final output,
    12. How to display the font size in use in the final output,


    Lipsum me [31-35]

\end{otherlanguage*}



% \end{apendicesenv}

% TODO - Adicionar anexos
% % Inicia os anexos
% \begin{anexosenv}
%     % Imprime uma página indicando o início dos anexos
%     \ifforcedinclude\else\partanexos\fi
%     \setlength\beforechapskip{50pt}
%     \setlength\midchapskip{20pt}
%     \setlength\afterchapskip{20pt}

%     

%
% How to fix the Underfull \vbox badness has occurred while \output is active on my memoir chapter style?
% https://tex.stackexchange.com/questions/387881/how-to-fix-the-underfull-vbox-badness-has-occurred-while-output-is-active-on-m
%

% ----------------------------------------------------------
\chapter{\lang{Article published in SOBRAEP magazine}{Artigo publicado}}
% ----------------------------------------------------------


% Multiple-language document - babel - selectlanguage vs begin/end{otherlanguage}
% https://tex.stackexchange.com/questions/36526/multiple-language-document-babel-selectlanguage-vs-begin-endotherlanguage
\begin{otherlanguage*}{english}

% An environment for setting \emergencystretch locally
% https://tex.stackexchange.com/questions/84510/an-environment-for-setting-emergencystretch-locally
{
    \setlength{\emergencystretch}{10pt}
    \section[English guidelines for publication]
    {English guidelines for publication - TITLE HERE (14 PT TYPE SIZE, UPPERCASE, BOLD, CENTERED)}
}
    \noindent\textbf{Abstract:}
    The objective of this document is to instruct the authors about the preparation of the
    manuscript for its submission to the Revista Eletrônica de Potência (Brazilian Power Electronics
    Journal).~The authors should use these guidelines for preparing both the initial and final
    versions of their paper. Additional information about procedures and guidelines for publication
    can be obtained directly with the editor, or through the web site
    \url{http://www.sobraep.org.br/revista}. This text was written according to these guidelines

\end{otherlanguage*}

% What is a “Overfull \hbox (9.89561pt too wide)”?
% https://tex.stackexchange.com/questions/111948/what-is-a-overfull-hbox-9-89561pt-too-wide
interwordspace: \the\fontdimen2\font

interwordstretch: \the\fontdimen3\font

emergencystretch: \the\emergencystretch\par\relax


\modifiedincludepdf{-}{ArtigoSOBRAEP}{pictures/SOBRAEP.pdf}{0.9}



%     


%
% How to fix the Underfull \vbox badness has occurred while \output is active on my memoir chapter style?
% https://tex.stackexchange.com/questions/387881/how-to-fix-the-underfull-vbox-badness-has-occurred-while-output-is-active-on-m
%

% ----------------------------------------------------------
\lang
{\chapter[Sample example]{How to display the font size in use in the final output}}
{\chapter[Anexo exemplo]{Como exibir o tamanho da fonte em uso na saída final}}
% ----------------------------------------------------------


% Multiple-language document - babel - selectlanguage vs begin/end{otherlanguage}
% https://tex.stackexchange.com/questions/36526/multiple-language-document-babel-selectlanguage-vs-begin-endotherlanguage
\begin{otherlanguage*}{english}

\showfont

1. How to display the font size in use in the final output,
2. How to display the font size in use in the final output,
3. How to display the font size in use in the final output,


\section[Some encoding tests]{\showfont}

1. How to display the font size in use in the final output,
2. How to display the font size in use in the final output,
3. How to display the font size in use in the final output,
4. How to display the font size in use in the final output,
5. How to display the font size in use in the final output,
6. How to display the font size in use in the final output,

7. How to display the font size in use in the final output,
8. How to display the font size in use in the final output,
9. How to display the font size in use in the final output,
10. How to display the font size in use in the final output,
11. How to display the font size in use in the final output,
12. How to display the font size in use in the final output,

\subsection{\showfont}

1. How to display the font size in use in the final output,
2. How to display the font size in use in the final output,
3. How to display the font size in use in the final output,
4. How to display the font size in use in the final output,
5. How to display the font size in use in the final output,
6. How to display the font size in use in the final output,

7. How to display the font size in use in the final output,
8. How to display the font size in use in the final output,
9. How to display the font size in use in the final output,
10. How to display the font size in use in the final output,
11. How to display the font size in use in the final output,
12. How to display the font size in use in the final output,

\subsubsection{\showfont}

1. How to display the font size in use in the final output,
2. How to display the font size in use in the final output,
3. How to display the font size in use in the final output,
4. How to display the font size in use in the final output,
5. How to display the font size in use in the final output,
6. How to display the font size in use in the final output,

7. How to display the font size in use in the final output,
8. How to display the font size in use in the final output,
9. How to display the font size in use in the final output,
10. How to display the font size in use in the final output,
11. How to display the font size in use in the final output,
12. How to display the font size in use in the final output,

\subsubsubsection{\showfont}

1. How to display the font size in use in the final output,
2. How to display the font size in use in the final output,
3. How to display the font size in use in the final output,
4. How to display the font size in use in the final output,
5. How to display the font size in use in the final output,
6. How to display the font size in use in the final output,
7. How to display the font size in use in the final output,

8. How to display the font size in use in the final output,
9. How to display the font size in use in the final output,
10. How to display the font size in use in the final output,
11. How to display the font size in use in the final output,
12. How to display the font size in use in the final output,


Lipsum me [55-65]

\end{otherlanguage*}



% \end{anexosenv}

\end{document}

