\chapter{Descrição completa da sintaxe de Ipe em EBNF}
\label{apendix:ebnf-syntax}

Neste apêndice, apresentamos a sintaxe completa de
Ipe usando a notação EBNF, de acordo com \cite{ebnfstandard} (como iremos fazer
o nosso próprio \textit{parser}-- ao invés de usar um gerador automático --a
gramática não será otimizada-- as ferramentas que usaremos para construir o
\textit{parser} fazem isso por nós --, e alguns detalhes não serão apresentados,
como espaços em branco e comentários de bloco entre tokens). O símbolo inicial
da gramática é \texttt{file}. Discutimos mais sobre a sintaxe de Ipe no
\autoref{chapter:specification}.

\begin{bnfgrammar}
    file ::= module, \{new type $\vert$ tl value\}-;
    ;;
    line comment ::= "\slash\slash", \{? qualquer caractere ?\} - "\textbackslash n", "\textbackslash n";
    ;;
    block comment ::= "\slash*", \{? qualquer caractere ?\} - "*\slash", "*\slash";
    ;;
    doc comment ::= "\slash$\vert$*", \{? qualquer caractere ?\} - "*\slash", "*\slash";
    ;;
    upper identifier ::= uppercase letter, \{letter $\vert$ digit $\vert$ "\_"\};
    ;;
    lower identifier ::= lowercase letter, \{letter $\vert$ digit $\vert$ "\_"\};
    ;;
    number ::= ["$-$"], \{digit\}-, [".", \{digit\}-];
    ;;
    string ::= "'", \{? qualquer caractere ?\} - ("\textbackslash n" $\vert$ "'"), "'";
    ;;
    module name ::= \{upper identifier, "."\}, upper identifier;
    ;;
    exported item ::= letter, \{letter $\vert$ digit $\vert$ "\_"\};
    ;;
    export list ::= "[", \{exported item\}-, "]";
    ;;
    module decl ::= "module", module name, "exports", export list;
    ;;
    import decl ::= "import", module name, [ "as", module name ];
    ;;
    module ::= module decl, [doc comment], \{import decl\};
    ;;
    field decl ::= lower identifier, "$\colon$", expression;
    ;;
    record ::= "\{", [field decl, \{",", field decl\}], "\}";
    ;;
    custom type ::= \{upper identifier, "."\}, upper identifier, \{any type\};
    ;;
    any type ::= custom type
    | lower identifier
    | record annotation;
    ;;
    record ann field ::= lower identifier, "$\colon$", any type;
    ;;
    record annotation ::= "\{", [record ann field, \{",", record ann field\}] "\}";
    ;;
    alias ::= upper identifier, parameters, "=", any type;
    ;;
    union ::= upper identifier, parameters, "=", \{constructor\}-;
    ;;
    constructor ::= "$\vert$", upper identifier, [record annotation];
    ;;
    new type ::= "type alias", alias
    | "type union", union
    | "type opaque", union;
\end{bnfgrammar}
\begin{bnfgrammar}
    fn args ::= \{lowercase identifier\};
    ;;
    function ::= "\textbackslash", fn args, "$-$>", \{attribution\}, expression;
    ;;
    tl annotation ::= lowercase identifier, "$\colon$", \{any type, "$-$>"\}, any type;
    ;;
    attribution ::= lowercase identifier, "$=$", expression, "$;$";
    ;;
    tl left ::= [doc comment], [tl annotation], lowercase identifier;
    ;;
    tl value ::= tl left "$=$", expression;
    ;;
    type destruct ::= \{upper identifier, "."\}, upper identifier, \{pattern\};
    ;;
    pattern ::= "\_"
    | number
    | string
    | type destruct;
    ;;
    match case ::= "$\vert$", pattern, "$-$>", \{attribution\}, expression;
    ;;
    match ::= "match", expression, "with", \{match case\}-;
    ;;
    expression ::= expression, "$\vert$>", expression
    | expression, "$<$$\vert$", expression
        | function
        | match
        | exponentiation;
        ;;
        exponentiation ::= exponentiation, "\textasciicircum", term
        | term;
        ;;
        term ::= term, ("+" $\vert$ "$-$"), factor
        | factor;
        ;;
        factor ::= factor, ("\slash" $\vert$ "*"), primary
    | primary;
    ;;
    primary ::= number
    | string
    | record
    | list
    | variable name, \{primary\}
    | "(", expression, ")";
    ;;
    list ::= "[", [expression, \{",", expression\}], "]";
    ;;
    variable name ::= \{uppercase identifier, "."\}, record access;
    ;;
    record access ::= lowercase identifier, ".", record access
    | lowercase identifier;
\end{bnfgrammar}
