

\newcommand{\imprimirbrazilabstract}{%
    \cleardoublepage\phantomsection
    \addtotextpreliminarycontent{Resumo em Português}
    \begin{otherlanguage*}{brazil}
        \begin{resumo}[Resumo]

            Grande parte dos produtos comerciais necessitam de um \textit{backend}.
            Este trabalho apresenta Ipe, uma linguagem de programação
            voltada especificamente para desenvolver sistemas \textit{backend},
            com APIs REST, seguindo o paradigma funcional. Com as garantias do
            paradigma funcional e da tipagem estática, Ipe busca ser uma linguagem
            simples e fácil de se usar, ao mesmo tempo em que tenta capturar erros
            em tempo de compilação, diminuindo erros em tempo de execução. Com um compilador
            escrito em Haskell, Ipe é compilada para código Javascript, e apresenta várias
            características desejáveis para o desenvolvimento de aplicações \textit{backend},
            como \textit{pattern matching}, funções puras, tipagem forte e estática, e
            inferência de tipos.

            \imprimirpalavraschave{Palavras-chaves}{\begin{inparaitem}[]\palavraschaveportugues\end{inparaitem}}

        \end{resumo}
    \end{otherlanguage*}
}


\newcommand{\imprimirenglishabstract}{%
    \cleardoublepage\phantomsection
    \addtotextpreliminarycontent{English's Abstract}
    \begin{otherlanguage*}{english}
        \begin{resumo}[Abstract]

            Many commercial products need a backend. This work presents Ipe, a programming language
            specifically designed to develop backend systems, with REST APIs, following the functional
            paradigm. With the guarantees of the functional paradigm and static typing, Ipe aims to
            be a simple and easy to use language, while trying to capture errors at compile time,
            reducing runtime errors. With a compiler written in Haskell, Ipe is compiled to
            Javascript, and presents several desirable features for backend development, such as
            pattern matching, pure functions, strong and static typing, and type inference.

            \imprimirpalavraschave{Keywords}{\begin{inparaitem}[]\palavraschaveingles\end{inparaitem}}

        \end{resumo}
    \end{otherlanguage*}
}


\imprimirbrazilabstract
\imprimirenglishabstract
