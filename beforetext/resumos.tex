

\newcommand{\imprimirbrazilabstract}{%
    \cleardoublepage\phantomsection
    \addtotextpreliminarycontent{Resumo em Português}
    \begin{otherlanguage*}{brazil}
        \begin{resumo}[Resumo]

            Grande parte dos produtos comerciais necessitam de um \textit{backend}.
            Este trabalho apresenta Ipe, uma linguagem de programação feita
            voltada especificamente para desenvolver sistemas \textit{backend},
            com APIs REST, seguindo o paradigma funcional. Com as garantias do
            paradigma funcional e da tipagem estática, Ipe busca ser uma linguagem
            simples e fácil de se usar, ao mesmo tempo em que tenta capturar erros
            em tempo de compilação, diminuindo erros em tempo de execução.

            \imprimirpalavraschave{Palavras-chaves}{\begin{inparaitem}[]\palavraschaveportugues\end{inparaitem}}

        \end{resumo}
    \end{otherlanguage*}
}


\newcommand{\imprimirenglishabstract}{%
    \cleardoublepage\phantomsection
    \addtotextpreliminarycontent{English's Abstract}
    \begin{otherlanguage*}{english}
        \begin{resumo}[Abstract]

            Many commercial products need a \textit{backend}. This work presents
            Ipe, a programming language made specifically to develop \textit{backend}
            systems, following the functional style. With the guarantees of the
            functional paradigm and static typing, Ipe seeks to be a simple and
            easy to use language, while trying to capture all errors at compile
            time, avoiding errors at runtime.

            \imprimirpalavraschave{Keywords}{\begin{inparaitem}[]\palavraschaveingles\end{inparaitem}}

        \end{resumo}
    \end{otherlanguage*}
}


\imprimirbrazilabstract
\imprimirenglishabstract
