\documentclass[12pt]{article}

\usepackage{sbc-template}
\usepackage{graphicx,url}
% \usepackage[utf8]{inputenc}
\usepackage[brazil]{babel}
% \usepackage[latin1]{inputenc}  
\usepackage{listings}
\lstset{
  numbers=left,
  stepnumber=1
}
\renewcommand{\lstlistingname}{Código}

     
\sloppy

\title{Linguagem de Programação Ipe: Programação Puramente Funcional para APIs REST}

\author{Henrique da Cunha Buss\inst{1}}


\address{Departamento de Informática e Estatística -- Universidade Federal de Santa Catarina
  (UFSC)\\
  \email{henrique.buss@grad.ufsc.br}
}

\begin{document}

\maketitle

\begin{abstract}
  This paper presents Ipe, a purely functional programming language, aimed at developing
  \textit{backend} applications for REST APIs. Ipe aims to be easy to learn and use, while providing
  a strong and static type system, with the intention of reducing runtime errors. We compare Ipe
  with other languages and frameworks available, to show that Ipe is a viable alternative for
  developing \textit{backend} applications, that tries to improve in several areas in relation to
  the alternatives.
\end{abstract}

\begin{resumo}
  Este trabalho apresenta Ipe, uma linguagem de programação puramente funcional, voltada ao
  desenvolvimento de aplicações \textit{backend} para APIs REST. Ipe tem como objetivo ser fácil de
  aprender e de usar, além de fornecer um sistema de tipos forte e estático, com a intenção de diminuir
  erros em tempo de execução. Comparamos Ipe com outras linguagens e \textit{frameworks} disponíveis,
  para mostrar que Ipe é uma alternativa viável para o desenvolvimento de aplicações \textit{backend},
  que tenta melhorar em várias áreas em relação às alternativas.
\end{resumo}


\section{Introdução}

Em contraste com o paradigma de orientação a objetos, o paradigma funcional visa simplificar o modelo
mental dos programadores, focando em funções puras, imutabilidade, e composição de funções \cite{functionalprogramming-future}.
Embora esses conceitos possam criar programas mais robustos e mais fácies de testar, pode ser difícil
para desenvolvedores acostumados com outros paradigmas a aprenderem esse novo modelo mental \cite{promisesoffp}.
Não existe uma definição exata para o que é uma linguagem puramente funcional, mas elas geralmente
têm alguns recursos em comum, como \textit{pattern matching}, funções puras (sem efeitos colaterais),
tipagem forte e estática, e inferência de tipos. Alguns exemplos de linguagens puramente funcionais
são \textit{Haskell} \cite{haskellintroduction} e \textit{Elm} \cite{czaplicki2012elm}.

Ipe é uma linguagem de programação puramente funcional, fortemente inspirada por Elm, com o objetivo
de ser fácil de aprender e de usar, enquanto fornece um sistema de tipos forte e estático, com a
intenção de reduzir erros em tempo de execução. Ipe é uma linguagem de programação desenvolvida
especificamente para o desenvolvimento de APIs REST, e por isso possui recursos que facilitam o
desenvolvimento de aplicações \textit{backend}.


\section{Comparação com outras linguagens} \label{sec:comparisons}

A tabela \ref{tab-language-comparisons} mostra características de algumas linguagens de programação
e de Ipe. As linguagens foram escolhidas com base na pesquisa anual do Stack Overflow \cite{stackoverflowsurvey}
de 2022, e o índice TIOBE \cite{tiobeindex}. A tabela mostra que Ipe é uma linguagem de programação
com características similares a Haskell, mas com o objetivo de ter um nível de temor menor.


\begin{table}[htb]
  \caption{Principais características das linguagens analisadas e de Ipe. O nível de temor, determinado por \cite{stackoverflowsurvey}, mede o quão temida é a linguagem.}
  \label{tab-language-comparisons}
  \begin{tabular}{p{2.50cm}p{5.00cm}p{2.50cm}p{3.25cm}}
    \textbf{Linguagem} & \textbf{Paradigma principal}             & \textbf{Tipagem} & \textbf{Nível de temor} \\
    Python             & OO                                       & Dinâmica         & 32,66\%                 \\
    C\#                & OO                                       & Estática         & 36,61\%                 \\
    Javascript         & OO com protótipos e elementos funcionais & Dinâmica         & 38,54\%                 \\
    Elixir             & Funcional                                & Dinâmica         & 24,54\%                 \\
    Haskell            & Puramente funcional                      & Estática         & 43,56\%                 \\
    Ipe                & Puramente funcional                      & Estática         & --                      \\
  \end{tabular}
\end{table}

Para a comparação das linguagens, foi desenvolvida uma aplicação base, que consiste em um sistema de
cadastro de usuários (um simples \textit{CRUD -- Create, Read, Update, Delete}). A aplicação deve receber
requisições HTTP para cadastrar, atualizar, deletar e listar usuários. Os dados devem ser enviados
através do corpo da requisição, em formato JSON, seguindo a convenção REST. Cada usuário deve ter um
\texttt{id} numérico, nome e idade. A tabela \ref{tab-base-endpoints} mostra todos os \textit{endpoints}
da aplicação base.


\begin{table}[htb]
  \caption[\textit{Endpoints} da aplicação base]{\textit{Endpoints} da aplicação base}
  \label{tab-base-endpoints}
  \begin{tabular}{p{2.85cm}p{2.85cm}p{7.55cm}}
    \textbf{Método HTTP} & \textbf{\textit{Endpoint}} & \textbf{Descrição}                                                                         \\
    GET                  & /users                     & Lista todos os usuários cadastrados                                                        \\
    GET                  & /users/:id                 & Lista um usuário específico                                                                \\
    POST                 & /users                     & Cria um novo usuário. O nome e a idade devem ser enviados no corpo da requisição           \\
    PUT                  & /users/:id                 & Atualiza um usuário específico. O nome e a idade devem ser enviados no corpo da requisição \\
    DELETE               & /users/:id                 & Deleta um usuário específico                                                               \\
  \end{tabular}
\end{table}

A aplicação desenvolvida em Python (usando o \textit{framework} Flask) possui um único arquivo, o
que possibilita prototipação rápida, e a complexidade inicial é baixa. Porém, Flask usa
\textit{decorators}, que podem ser confusos para iniciantes. A solução em C\#, também faz uso de
\textit{decorators}, além de organizar o código em vários arquivos, o que pode ser confuso para
iniciantes, embora a maioria tenha sido gerado automaticamente por ferramentas de geração de código.
C\# utiliza classes e métodos dessas classes para definir os \textit{endpoints}. A solução em
Javascript (usando o \textit{framework} Express) é similar à solução em Python, mas sem \textit{decorators}.
A desvantagem de Javascript é que endpoints são definidos com \textit{strings}, que devem ser
mantidas consistentes com os nomes das variáveis acessadas nos parâmetros da requisição. Elixir também
utiliza bastante geração de código com o \textit{framework} Phoenix. Embora Elixir seja uma linguagem
funcional, ela não possui um sistema de tipos estático, o que pode levar a erros em tempo de execução.
Por fim, Haskell, a única linguagem funcional com sistema de tipos estático da lista, também faz forte
uso de ferramentas de geração de código, e possui uma curva de aprendizado muito íngreme, o que pode
ser um problema para iniciantes.

A partir destes experimentos, podemos concluir que, se Ipe tem como objetivo ser uma linguagem simples,
de fácil aprendizado, devemos evitar \textit{decorators} e ferramentas de geração de código. Além disso,
Ipe deve possuir um sistema de tipos estático, para evitar erros em tempo de execução. A solução
proposta para Ipe é utilizar \textit{pattern matching} para definir os \textit{endpoints}, e
utilizar \textit{encoders} e \textit{decoders} para interagir com dados do mundo externo (JSON). Além
disso, para manter a simplicidade, é importante que Ipe ofereça um único jeito de fazer cada coisa,
para evitar confusão.

\section{Especificação da Linguagem}

Programas Ipe são compostos por módulos, onde cada módulo representa um arquivo. Arquivos Ipe têm a
extensão \texttt{.ipe}, ou seja, um arquivo chamado \texttt{Main.ipe} representa o módulo \texttt{Main}.
Cada módulo pode exportar funções e tipos, que podem ser utilizadas por outros módulos.

O código \ref{lst:language-example} mostra um exemplo de código Ipe. Nele, podemos ver que todos os
módulos começam com a sua definição, junto com as funções e tipos que o módulo exporta. Ipe suporta
comentários de linha, bloco e documentação, que são ignorados pelo compilador, mas podem ser usados
para outras ferramentas de desenvolvimento. Ipe também suporta importação de módulos, que podem ser
usados para reutilizar código. Um dos pontos fortes de Ipe é seu sistema de tipos, que permite que o
programador defina novos tipos, através dos comandos \texttt{type union}, \texttt{type opaque} e
\texttt{type alias}. Uniões de tipo (como mostrado nas linhas 8 a 10) possibilitam a modelagem de
problemas complexos, e são uma das principais ferramentas de Ipe. Ipe também suporta a definição de
funções, que podem ter uma anotação de tipo (linha 12), embora não seja necessário, pois Ipe consegue
inferir os tipos de qualquer programa válido. Além disso, Ipe também tem suporte a \textit{pattern matching},
como mostram as linhas 14 a 17.

\begin{lstlisting}[label={lst:language-example},caption={Exemplo de  código Ipe}]
module Pasta1.Pasta2.Modulo exports [ sayHello ]

// Comentario de linha
/* Comentario de bloco */

import Database.User as DbUser

type union User =
  | Guest
  | LoggedIn { name : String }

sayHello : User -> String
sayHello =
  \user ->
    match user with
      | Guest -> 'Hello there!'
      | User user -> String.concat [ 'Hello, ', user.name ]
\end{lstlisting}

\section{Processo de compilação}

O compilador Ipe, escrito em Haskell, é dividido em 4 partes: análise sintática e léxica,
transformação, checagem de tipos, e emissão de código. A etapa de análise sintática e léxica, também
conhecida como \textit{parsing} \cite{dragonbook}, é responsável por transformar texto que representa
um programa Ipe em um formato de árvore, que representa a gramática de Ipe.

Na etapa seguinte, de transformação de código, o compilador normaliza a árvore gerada na etapa anterior,
para facilitar as etapas seguintes. No futuro, algumas otimizações, como substituições de variáveis
por constantes, poderiam ser realizadas nessa etapa. A etapa de checagem de tipos é responsável por
garantir que todos os tipos das funções se alinham, e que não há erros de tipos. Esta etapa é a que
garante a maior parte da segurança de Ipe. O algoritmo de checagem de tipos é baseado em uma adaptação
do algoritmo W \cite{understaingalgorithmw}, com adição de uniões de tipos, tipos opacos,
\textit{records} e listas. Nesta etapa também é realizada a inferência de tipos de funções que não
possuem anotações de tipos. Na última etapa, de emissão de código, o compilador transforma cada folha
da árvore Ipe em trechos de código Javascript, e vai juntando os nós, das folhas à raiz.

Para começar o processo de compilação, basta invocar o aplicativo de linha de comando de Ipe,
informando o arquivo de entrada, que deve conter uma função \texttt{main}. O compilador de Ipe apenas
irá gerar código para o módulo de entrada, e os módulos importados (direta ou indiretamente) por ele.
Desta forma, módulos que nunca são importados não serão compilados. Como etapa final do processo de
compilação, o compilador baixa as bibliotecas padrão de Ipe, que são apenas interfaces (funções e tipos)
que representam as funções e tipos nativos de Javascript. Estas interfaces são utilizadas para
realizar a checagem de tipos, e são removidas do código final. Para executar o programa resultante,
basta utilizar algum \textit{runtime} Javascript, como Node, Deno ou Bun. O código Javascript
de algumas bibliotecas padrão (como \texttt{Http}) utiliza funções oferecidas pelo Bun, portanto
é recomendável utilizar Bun para executar programas Ipe.

\section{Resultados obtidos}

Embora o código em Ipe que representa uma solução à aplicação base, proposta na seção \ref{sec:comparisons},
seja a solução com maior número de linhas de código (sem considerar soluções geradas automaticamente),
Ipe fornece maior segurança na validação dos dados, através de \textit{encoders} e \textit{decoders}
JSON, além de maior simplicidade na definição de rotas, que usa \textit{pattern matching} para extrair
variáveis, ao invés de definir uma \textit{string} que define a rota, ou simplesmente receber como
parâmetro em uma função. Com estes recursos, somos forçados a pensar sobre casos de erro, e lidar com
casos em que os dados recebidos não possuem o formato esperado. Além disso, a estratégia de roteamento
com \textit{pattern matching} possibilita a implementação de \textit{middlewares}, que podem agilizar
o desenvolvimento. No caso de Ipe, basta definir uma função, e chamá-la dentro da função que define
o tratamento de rotas.

\section{Conclusões}

Ipe é uma linguagem de programação funcional, que busca ser simples e fácil de se usar, ao mesmo
tempo em que tenta capturar erros em tempo de compilação, diminuindo erros em tempo de execução. Ipe
é uma linguagem que busca ser útil no mundo real, e que pode ser usada para criar aplicações backend
modernas, com fortes garantias de tipos e sem efeitos colaterais. Ipe é uma linguagem que busca ser
fácil de se aprender, por ter poucos conceitos e uma sintaxe simples.

Desta forma, Ipe apresenta vantagens sobre linguagens como Javascript e Python, que possuem tipagem
fraca e dinâmica, e também apresenta vantagens sobre linguagens como Haskell, que, embora possuam um
sistema de tipos forte e estático, possuem uma alta barreira de entrada. Ipe também precisa de menos código do que frameworks
como ASP.NET, que se baseiam em ferramentas de geração automática de código, o que pode facilitar a
manutenção de aplicações Ipe. Por outro lado, Ipe ainda é uma linguagem nova, e precisa de mais
testes e mais aplicações reais para que possamos ter certeza de que ela cumpre seus objetivos. Além
disso, Ipe ainda precisa de mais ferramentas, como integração com editores de código, bibliotecas de
teste, e um gerenciador de pacotes, para que a comunidade possa compartilhar código mais facilmente.
Ipe também não apresenta otimizações de código, o que pode ser um problema para aplicações backend,
que muitas vezes precisam de alta performance para atender várias requisições simultaneamente.


\bibliographystyle{sbc}
\bibliography{../aftertext/references}

\end{document}
